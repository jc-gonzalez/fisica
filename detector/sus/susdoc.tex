\documentclass[a4paper,10pt]{article}

\input{mypapers}

%%%%%%%%%%%%%%%%%%%%%%%%%%%%%%%%%%%%%%%%%%%%%%%%%%%%%%%%%%%%
%% BEGIN
%%%%%%%%%%%%%%%%%%%%%%%%%%%%%%%%%%%%%%%%%%%%%%%%%%%%%%%%%%%%

\begin{document}

%% title %%%%%%%%%%%%%%%%%%%%%%%%%%%%%%%%%%%%%%%%%%%%%%%%%%%

\title{On the policies for submission of new revisions of source code
  files}

\author{J.C.Gonz\'alez\\
Max-Planck-Institut f\"ur Physik, Munich\\
{\tt <gonzalez@mppmu.mpg.de>}}

\date{8/3/99}

\maketitle

\begin{abstract}
  In this document we make a review of different, possible committing
  policies related with source code files. The source code revisions
  management is crucial in a big project, and therefore it has to be
  defined when and how the new revisions of the various modules of the
  projects have to be done. We will also talk about the releases
  policies, which is even more important, since each release of an
  application is meant to be (hopefully) a working version for some time.
\end{abstract}

%% body %%%%%%%%%%%%%%%%%%%%%%%%%%%%%%%%%%%%%%%%%%%%%%%%%%%%%

%------------------------------------------------------------
\section{Introduction}

It is clear for everybody that the use of any kind of version control
system (hereafter VCS) is very important for any project involving
either a large amount of separated modules (source code files,
libraries, documentation, ... ) or a group of people working together,
or both. It is even more important when all these people are working
in different places, and an easy way to follow in order not to run
into troubles is to use a VCS (plus fluid comunication between the
members of the project).

Typical tasks realized with the help of a VCS tool are: getting a
given set of files in order to use (compile), update, check, modify
and/or correct them; getting a previous release of a set of files in
order to compare it with a later version; develope a different
\emph{branch} in the source code versions tree, perhaps to test
improvements in algorithms, or to correct bugs; perform periodical
upgrades of the logging files, in order to have a clear idea of how
the development is going on; produce fixed releases of a whole
application, without interrupting the \emph{development process} in
another branch of the project; and, finally, allow users to work
without mayor difficulties all together in the same piece of code,
with a minimal risk of collision.

We first should have a clear idea of what a VCS is, and get familiar
with the usual nomenclature used.

%------------------------------------------------------------
\section{Versions, revisions, releases}


%------------------------------------------------------------
\section{Revisions policies}

\subsection{Conservative approach}

\subsection{Liberal approach}

%------------------------------------------------------------
\section{Releases policies}

%------------------------------------------------------------
\section{Conclusions}


\tableofcontents

\end{document}
