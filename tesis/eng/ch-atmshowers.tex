%%%%%%%%%%%%%%%%%%%%%%%%%%%%%%%%%%%%%%%%%%%%%%%%%%%%%%%%%%%%%%%%%%%%%%%%%%%
%%
%%  ch-atmshowers.tex
%%
%%  Created: Fri Oct 10 14:24:37 1997
%%  Author.: Jose Carlos Gonzalez
%%  Notes..:
%%          
%%-------------------------------------------------------------------------
%% Filename: $RCSfile$
%% Revision: $Revision$
%% Date:     $Date$
%%%%%%%%%%%%%%%%%%%%%%%%%%%%%%%%%%%%%%%%%%%%%%%%%%%%%%%%%%%%%%%%%%%%%%%%%%%


\chapter{Atmospheric gamma and cosmic-ray showers}
\label{chapter:atmshowers}

The particles that conform the CRAD, when entering into the Earth
atmosphere, interact with molecules and ionized atoms, producing
secondary particles. These secondary particles interact themselves and
with another molecules and atoms. This process, repeated along several
generations, is the so called \emph{\I{atmospheric shower}} (see Fig.
\ref{fig:atmshower}).

Obviously, different cosmic particles will generate different kinds of
atmospheric showers. In particular, if the incident particle
(\emph{\I{primary particle}} or, abbreviated,
\emph{\Isee{primary}{primary particle}} from now on) is a
$\gamma$-ray, the composition of the atmospheric shower will be mostly
electromagnetic (photons, electrons, positrons, muons, \ldots). We
will call this kind of shower \emph{\Isee{electromagnetic atmospheric
    shower}{EAS}}, or \emph{\I{EAS}}. On the contrary, if the primary
is an atomic nucleus (hadron), the dominant component of the shower
will be hadronic (and we will call it \emph{\Isee{hadronic atmospheric
    shower}{HAS}}, or \emph{\I{HAS}}). Apart from having a different
primary, both kinds of showers are not really very well
differentiated: for instance, a $\pi^{0}$ generated in an interaction
nucleus--nucleus will desintegrate in a pair $\gamma$-$\gamma$, giving
us an electromagnetic sub-shower.

\atmshowerfig    % artistic view of a Cherenkov experiment (HEGRA?)

\processesfig  % processes involved in generation of gamma rays

The detection of CRAD from ground-based observatories is, as deduced
from everything said before, an indirect measurement. The detectors at
sea level, or even on top of high mountains, cannot observe the
original primary particles as they arrive to the Earth. Instead, they
only can detect the subparticles generated in the development of the
shower in the atmosphere. In this sense, the atmosphere is a huge
``calorimeter''. The most important consequence of this is that the
energy of the primary has to be inferred from indirect measurements,
such as the total number of particles of a given specie (like
electrons, muons, photons, \ldots), or their particular distribution
in the ground. This is of course a problem, since the less precise our
calculation of the energy of the primary particles, the more difficult
our job when trying to discriminate between the different models or
scenarios for their production and transport until the Earth.  It is
clear therefore that the main goal of our studies must be the search
for criteria to obtain in a precise and reliable way the energy of the
primary particles. In the way looking for this goal, the simulation by
computer of atmospheric showers is of the most importance.

%%%%%%%%%%%%%%%%%%%%%%%%%%%%%%%%%%%%%%%%%%%%%%%%%%%%%%%%%%%%
\section{Development of Atmospheric showers} 

Before entering in the details of the simulation, let's study a little
bit more the physics involved in the development of an atmospheric
shower.

%%==========================================================
\subsection{Electromagnetic atmospheric showers}

\toymodelfig
%
The physical processes involved in the development of an atmospheric
showers are shown in Fig. \ref{fig:processes}. A gamma ray or a lepton
(electron or positron) can produce an EAS by bremsstrahlung and
pair-production. By these means, an EAS can evolve from the primary to
a huge amount of particles. Among them, the charged particle generate
also Cherenkov photons.

Usually, we start studying the physics behind EAS using the so called
\emph{toy-model}. This model is used to explain the development of an
EAS, but its basic structure can be applied to HAS as well. The model
assumes that a gamma ray $\gamma$ of energy $E_0$, when entering in
the atmosphere, travels a distance $\lambda$ before creating a pair
e$^+$e$^-$. Each of them takes half of the energy of the former
particle, that is $E_0/2$. At this first interaction we have 2
particle, each with half of the original energy.  These two particles
travel again a distance $\lambda$ until they suffer bremsstrahlung and
produce two photons (one per particle) of energy half of the energy of
that particle. After this second process we have 4 particles, each
with a mean energy of $E_0/4$. Following the development, after $n$
branchings the shower has traveled along a distance $X=n\lambda$
(where $X$ is the slant depth along the shower axis), and we will have
%
\toyAeq
%
particles with a mean energy of
%
\toyBeq

The whole process is shown in Fig.  \ref{fig:toymodel}.  Both
bremsstrahlung and pair production (or whatever splitting process we
assume, for instance, in a HAS) continue until
%
\toyCeq
%
$E_{\mathrm{c}}$ being the critical energy where these processes are
dominated by ionization and Compton scattering ($E_{\mathrm{c}} \sim
102 \u{MeV}$ in the atmosphere). The number of particles at shower
maximum in this model (reached when the mean energy of the particles
in the shower is $E\simeq E_{\mathrm{c}}$) is
%
\toyDeq

\hadronicfig

The basic features of these equations hold both for high energy EAS
and also for HAS, namely
%
\NXsimpleeq

%%==========================================================
\subsection{Hadronic showers}

The basic behaviour of a HAS can be outlined with the same model used
for EAS. An schematic diagram of the development of a nucleonic
cascade in the atmosphere is shown in Fig. \ref{fig:hadronic}.

However, for a nucleonic primary we have to know how showers generated
by heavy nuclei differ from those generated by protons or light
nuclei. For modelling nucleonic primaries the simplest model is the
\emph{superposition model}. In this model it's assumed that a nucleus
of mass $A$ and total energy $E_0$ is equivalent to $A$ independent
nuclei each of energy $E_0/A$. In this model we still have
$N(X_{\mathrm{max}}) = E_0 / E_{\mathrm{c}}$, but \eqref{eq:xmax}
becomes
%
\NXsimpleHadeq
%
In reality what happens is that a heavy nucleus interacts with the
atoms and molecules of the atmosphere, and only few nucleons are
released and produce pions and other particles; the rest of the
nucleus is fragmented in smaller pieces, with one of them being much
bigger than the others. 

\lognefig

%%==========================================================
\subsection{Particles in the atmospheric shower}

Actually, a more detailed study of the development of the shower is
needed. Using Monte Carlo methods one can see that the mean
longitudinal profile of an EAS can be approximated by
%
\Neeq
%
where $N_{\mathrm{e}}$ is the number of electrons in the shower
(``size of the shower''), $\tau$ is the atmospheric depth in
radiation-lengths, $x=\ln(E_0/E_{\mathrm{c}})$ and $s$ is the
\emph{age parameter} of the shower:
%
\ageeq
%
The \emph{age parameter} $s$ changes its value from 0 at the
beginning of the shower to 1 at the maximum development, and the to 2
where the number of particles is below 1. In Fig. \ref{fig:logne} we
can see the theoretical longitudinal development of an EAS.

\samplelonglatdistfig

For the average lateral distribution of EAS we can use the so called
\emph{NKG formula}
%
\NKGeq
%
where $\rho(r)$ is the density of electrons as a function of the
distance to the shower axis $r$, $r_{\mathrm{M}}$ is the
\emph{Moli\`ere radius} of multiple scattering ($\sim 79\u{m}$ at sea
level, $\sim 106\u{m}$ at 2200\u{m} above sea level), and
%
\fsreq
%
The general form of the NKG formula has proved to be very useful in
fitting the lateral distributions obtained in Monte Carlo
simulations. In Fig. \ref{fig:samplelonglatdist} we can see an example of
mean longitudinal and lateral distributions for Monte Carlo generated
showers at 100\u{GeV} and 1\u{TeV} (primary energies) for an EAS.

%%%%%%%%%%%%%%%%%%%%%%%%%%%%%%%%%%%%%%%%%%%%%%%%%%%%%%%%%%%%
\section{Production of \Cherenkov light}

Cherenkov light is emitted when a charge particle travels through a
dielectric medium and its velocity is larger than the phase velocity
of the light in that medium. The condition is
%
\Cherenkovcondeq
%
where $\beta=v/c$, the particle velocity with respect to that of light
in the, and n is the refractive index of the dielectric medium
(proportional to the density of the medium). This condition can be
also expressed in terms of energy: a charge particle with an energy
above the minimum threshold energy $E_{\mathrm{min}}$ will produce 
polarization in the medium. The medium emits pulses which interfere,
and a front is produced (see Fig. \ref{fig:cherenkov1}). The minimum
threshold energy $E_{\mathrm{min}}$ varies with the height, the reason
being the variation of the refractive index. The light is emitted at
an angle $\theta$ with respect to the particle trajectory. The
radiation is strongly peaked at an angle $\theta_{\mathrm{c}}$, given
by the condition for coherence of the radiated light
%
\lightcoherenceeq
%
At sea level, $n \simeq 1.00029$ and $\theta_{\mathrm{c}} \simeq
1.3\deg$. From Eq. \eqref{eq:cherenkovcond} we can calculated the values
of $E_{\mathrm{min}}$ for \Cherenkov light emission in air by several
particles: for electrons is $21\u{MeV}$, for muons $4.6\u{GeV}$ and
for protons $39\u{GeV}$.

In order to understand a little bit more how the \Cherenkov front
evolves, let us use first a simplified model of atmosphere, the so
called \emph{standard atmosphere}. In this model, the density varies
exponentially with the height
%
\exprhoeq
%
where $H_{\mathrm{S}} \simeq 7.5\u{km}$ is the scale-height of the
atmosphere. The \emph{atmospheric thickness} $\tau$ is a measure of
the total mass between to points (normally from the top of the
atmosphere till a given altitute). It is calculated by integrating the
density along a given path. Therefore, in our model
%
\pathinteq

\cherenkovfig
%
The index of refraction of the medium can be expressed as 
%
\refractionidxeq
%
with $\tau_0=1030\u{g/cm^2}$ is the thickness at sea level and $T=204
+ 0.091\tau$ is the temperature at a given thickness. Therefore,
approximately
%
\etaeq
%
with $\eta_0 \simeq 0.00029$. In this model, the \Cherenkov angle
results 
%
\Cherenkovangleeq
%
and the minimum threshold $E_{\mathrm{min}}$
%
\Emineq

Now we are interested in the amount of Cherenkov light produced. If we
neglect, in first order approximation, the effects of Coulomb
scattering, we can write the loss of energy $E$ per unit path length
by a particle of charge $e$ and velocity $\beta$ to \Cherenkov
radiation of wavelength between $\lambda$ and $\lambda+\d \lambda$ as
%
\dEdheq
%
or, in terms of number of photons
%
\dNdheq
%
where $\alpha = e^2/\hbar c \simeq 1/137$ is the fine structure
constant, and $n$ is assumed constant over the integration interval.
We can see from this the general behaviour of the spectral
distributions:
%
\specdistreq
%
The number of photons emitted by an electron within a spectral region
$(\lambda_1,\lambda_2)$ in a path of length $l$ is then
%
\phemiteq
%
where we have used the condition of coherence of the Cherenkov light
\eqref{eq:lightcoherence}. The dependence $\lambda^{-1}$ implies that
most of the photons will have wavelengths in the UV domain.

The relation obtained in \eqref{eq:cherenkovlight} does not mean that
we can get an virtual infinite number of \Cherenkov photons: in fact,
the \Cherenkov light, generated in the UV range, is strongly absorbed
at wavelengths below 300\u{nm}. The main effects responsible for this
absorption are:

\begin{description}
  
\item[\textbf{Rayleigh scattering}] The molecules in the atmosphere
  produced scattering of the photons, being the effect of this
  scattering strongly dependent of the wavelength $\lambda$ of the
  light (more important for small wavelengths).
  
\item[\textbf{Mie scattering}] This effect is almost constant over the
  whole range of optical wavelengths. It is due to the aerosol (dust)
  particles suspended in the air. Its effect is bigger at lower
  heights. It depends also on the size of these particles.
  
\item[\textbf{Ozone absorption}] This effect is very important in the
  range 280--340\u{nm} and for low energy showers, in which most of
  the \Cherenkov light emission is located and the density of ozone is
  high.

\end{description}

%\begin{description}

%\item[\textbf{Rayleigh scattering}] The molecules in the atmosphere
%  produced scattering of the photons. This effects is strongly
%  correlated with the wavelength $\lambda$ of the light, being more
%  important for small wavelengths. The transmission coefficient due to
%  Rayleigh scattering is
%                                %
%  \Rayleigheq
%                                %
%  where $\tau_i = \tau_0 \exp(-h_i/H_{\mathrm{S}}) \sec\theta$ is the
%  \emph{slanted thickness} above a height $h_i$,
%  $\tau_{\mathrm{R}}(\lambda=400\u{nm}) = 2970\u{g/cm^2}$ is the mean
%  free path of the Rayleigh scattering in terms of thickness, $\tau_0
%  = 0.00129\u{g/cm^2}$, and $H_{\mathrm{S}} \simeq 7.5\u{km}$ is the
%  mentioned scale-height of the atmosphere.
  
%\item[\textbf{Mie scattering}] This effect is almost constant over the
%  whole range of optical wavelengths. It is due to the aerosol (dust)
%  particles suspended in the air, being its effect bigger at lower
%  heights. It depends also on the size of these particles. The
%  transmission coefficient is
%                                %
%  \Mieeq
%                                %
%  where $l_{\mathrm{M}}(400\u{nm}) \simeq 14\u{km}$ is the mean free
%  path for the Mie scattering, and $h_{\mathrm{M}} \simeq 1.2\u{km}$ is
%  the scale-height for the aerosol distribution.
  
%\item[\textbf{Ozone absorption}] This effect is very important in the
%  range 280--340\u{nm} and for low energy showers, in which most of
%  the \Cherenkov light emission is located where the density of ozone
%  is high.

%\end{description}

\endinput
%
%% Local Variables:
%% mode:latex
%% End:
%%EOF
