%%%%%%%%%%%%%%%%%%%%%%%%%%%%%%%%%%%%%%%%%%%%%%%%%%%%%%%%%%%%%%%%%%%%%%%%%%%
%%
%%  ch-cosmicrad.tex
%%
%%  Created: Fri Oct 10 14:24:37 1997
%%  Author.: Jose Carlos Gonzalez
%%  Notes..:
%%          
%%-------------------------------------------------------------------------
%% Filename: $RCSfile$
%% Revision: $Revision$
%% Date:     $Date$
%%%%%%%%%%%%%%%%%%%%%%%%%%%%%%%%%%%%%%%%%%%%%%%%%%%%%%%%%%%%%%%%%%%%%%%%%%%

\chapter{La Radiaci'on C'osmica}
\label{chapter:cosmicrad}
\let\rightmark\leftmark

\section{Origen de la Radiaci'on C'osmica}

La {\sl Radiaci'on C'osmica} ({\sl RAC}) es un ba~no de part'iculas
subat'omicas que puede encontrarse por todo el cosmos\footnote{En este
  trabajo hablaremos de {\sl Radiaci'on C'osmica} cuando queramos
  hablar tanto de part'iculas materiales como de cuantos de luz
  (fotones), y utilizaremos los t'erminos \emph{Rayos C'osmicos} y
  \emph{Rayos {\rm o} Radiaci'on Gamma} cuando nos refiramos
  exclusivamente y por separado a part'iculas materiales o bien a
  fotones, respectivamente.}.  Est'a formada por n'ucleos de 'atomos
ionizados y fotones de alta energ'ia. El grueso de los \emph{Rayos
  C'osmicos} ({\sl RC}), hasta un 90\%, son protones, siendo un 9\%
part'iculas $\alpha$ (n'ucleos de Helio) y el 1\% restante n'ucleos
pesados. Se miden en la Tierra a un ritmo aproximado de
$1\,000\u{s^{-1}\,m^{2}}$, y sus energ'ias pueden alcanzar los
$100\u{EeV}$ ($10^{20}\u{eV}$)\footnote{Sobre unidades y ordenes de
  magnitud, ver apendice \ref{ap:units}.}. Energ'ias tan altas s'olo
pueden obtenerse en escenarios astrof'isicos que involucren campos
electromagn'eticos y gravitatorios extraordinariamente intensos.

Existen varias posibles clasificaciones para las fuentes de RAC.
Pueden dividirse en gal'acticas o extragal'acticas, conforme a la
intensidad de emisi'on en distintos rangos energ'eticos, o bien seg'un
sean extensas o puntuales. Nosotros adoptaremos este 'ultimo criterio.
A continuaci'on repasaremos brevemente las caracter'isticas de los
candidatos o fuentes reconocidas m'as importantes.

\section{La F'isica de la Radiaci'on C'osmica}

\subsection{Generaci'on de la Radiaci'on C'osmica}

\subsubsection{Rayos C'osmicos cargados}

\subsubsection{Rayos Gamma}

\subsubsection{Neutrinos}

\subsubsection{Espectro de la Radiaci'on C'osmica}

\subsubsection{Composici'on qu'imica de la Radiaci'on C'osmica}

\subsection{Fuentes de Rayos C'osmicos}

La mayor parte del esfuerzo realizado en los que hoy d'ia denominamos
\emph{Astronom'ia de Rayos Gamma} se ha centrado en la busqueda y
posterior estudio de fuentes puntuales de RAC. Varias fuentes han sido
ya identificadas como emisoras de rayos gamma, pero su n'umero decrece
significativamente cuando incrementamos la energ'ia de las part'iculas
producidas. Entre las fuentes puntuales de radiaci'on c'osmica podemos
destacar los \emph{remanentes de supernova}, los \emph{p'ulsares}, los
\emph{n'ucleos de galaxias activos}, los \emph{cu'asares} y los
\emph{gamma-ray bursts} (literalmente, \emph{estallidos de
rayos gamma}).
 
\subsubsection{P'ulsares}
%
El sat'elite SAS\,II (\emph{small astronomical satellite}, peque~no
sat'elite astron'omico), lanzado el 15 de Noviembre de 1\,972,
suministr'o gran cantidad de datos sobre la emisi'on gamma
gal'atica. Entre sus resultados se encuentra el hallazgo de dos picos
en intensidad de flujo de rayos gamma correspondientes a dos p'ulsares
conocidos, Vela (PSR\,0833-45) y el pulsar del Cangrejo
(PSR\,0531+21). Su identificaci'on se vi'o confirmada al encontrarse
emisi'on pulsada de per'iodos semejantes a los de la emisi'on en
radio.

Recientemente otro pulsar ha sido confirmado como fuente de rayos
gamma: PSR\,1706-44. Hay autores que le atribuyen un remanente de
supernova, pero no es opini'on generalizada. En cualquier caso, s'i
parece tener asociada una nebulosa de rayos X.

Como regla general, los p'ulsares poseen caracter'isticas
dispares. Esto es cierto tambien para el pulsar del Cangrejo y
Vela. En el primero, por ejemplo, las curvas de luz son muy semejantes
en todos los rangos energ'eticos, desde el radio hasta algunos
GeV. Tambi'en se observa el caracter'istico pulso principal junto con
un interpulso m'as o menos importante. Por el contrario, en la curva
de luz de Vela correspondiente a la emisi'on en radio no existe tal
interpulso. Adem'as, las diferencias de fase
$\Phi(\mathrm{pulso})$--$\Phi(\mathrm{interpulso})$ son distintas en
rayos gamma y en el 'optico. Esto es un reflejo de los diferentes
mecanismos responsables de la emisi'on electromagn'etica para cada
pulsar.

\pulsarschfig

Existen diversos modelos que tratan de explicar la generaci'on de
rayos $\gamma$ en los p'ulsares. En el \emph{polar cap model} (modelo
de la capa polar) los rayos $\gamma$ son producidos por part'iculas
cargadas aceleradas cerca de los polos. Este modelo prev'e una
energ'ia m'axima para los rayos $\gamma$ generados de
$E=6-20\u{GeV}$. Si se producen por la aceleraci'on de electrones y
protones en las cercan'ias del \emph{cilindro de luz}\footnote{Se
define como \emph{cilindro de luz} la regi'on en las inmediaciones del
p'ulsar en el cual la velocidad de las lineas magn'eticas en
corrotaci'on con el pulsar alcanza el valor de la velocidad de la
luz}, donde pueden encontrarse discontinuidades en el campo
magn'etico, sus energ'ias podr'ian alcanzar los $100\u{GeV}$ (como
predice el \emph{outer cap model}, modelo de la capa externa).  Otros
modelos hacen uso del \emph{efecto Compton inverso}.  Fotones de
energ'ias en el rango de los rayos X podr'ian ser inyectados en el
dominio de los rayos $\gamma$ por colisiones con electrones
energ'eticos. En cualquiera caso, parece m'as que probable que las
emisiones en los diferentes rangos energ'eticos tengan or'igenes
distintos.

Todo lo dicho hasta ahora se refiere a pulsares activos en radio
(\emph{radio-pulsars}). Existen tambi'en p'ulsares sin emisi'on en
radio (\emph{radio-quiet pulsars}), entre los cuales solamente uno ha
sido identificado como fuente de rayos $\gamma$. Es el caso de
Geminga. Esta fuente discreta de rayos $\gamma$ ha suscitado desde el
principio gran expectaci'on, debido a que hasta recientemente no se le
ha podido atribuir contrapartida en alg'un otro rango energ'etico. En
la actualidad, sin embargo, parece claro que Geminga es un sistema
binario formado por una estrella de neutrones girando en torno a otra
estrella de neutrones mayor en r'apida rotaci'on.

\subsubsection{Supernovas}
%
Hasta ahora, incluso tras grandes esfuerzos por parte de los
experimentos operativos en el campo de la astronom'ia de rayos gamma,
s'olo tres remanentes de supernova han sido confirmados como fuentes
de radiaci'on gamma: la nebulosa del Cangrejo, Vela y SN\,1006, cuyo
descubrimiento como fuente gamma tuvo lugar en 1,997. Siguen
realiz'andose numerosas b'usquedas, pero hasta ahora solamente se han
podido obtener resultados en forma de l'imites superiores para los
flujos de rayos $\gamma$.

\subsubsection{N'ucleos de Galaxias Activos(AGNs)}
%
Pues eso, sobre los n'ucleos de Galaxias Activos

\subsubsection{Cu'asares}
%
Pues eso, sobre los cu'asares.

\subsubsection{\emph{Gamma Ray Bursts}}
%
Pues eso, sobre los \emph{gamma-ray bursts}.

\subsubsection{Fuentes extensas}

\subsection{Propagaci'on a trav'es de los medios 
interestelar e intergal'actico}

\subsubsection{Difusi'on en las regiones de formaci'on}

\subsubsection{Absorci'on en el medio intergal'actico}

\subsubsection{Absorci'on en el medio interestelar}

\subsection{Los Radiaci'on C'osmica a su entrada en la
atm'osfera terrestre}

\subsubsection{Generaci'on de cascadas atmosf'ericas}

\subsubsection{Frente de part'iculas}

\subsubsection{Frente de luz \Cherenkov}

\subsubsection{Observables de las cascadas atmosf'ericas}

\endinput
%
%%EOF
