%%%%%%%%%%%%%%%%%%%%%%%%%%%%%%%%%%%%%%%%%%%%%%%%%%%%%%%%%%%%%%%%%%%%%%%%%%%
%%
%%  ch-simmagic.tex
%%
%%  Created: Fri Oct 10 14:24:37 1997
%%  Author.: Jose Carlos Gonzalez
%%  Notes..:
%%          
%%-------------------------------------------------------------------------
%% Filename: $RCSfile$
%% Revision: $Revision$
%% Date:     $Date$
%%%%%%%%%%%%%%%%%%%%%%%%%%%%%%%%%%%%%%%%%%%%%%%%%%%%%%%%%%%%%%%%%%%%%%%%%%%
%

\chapter{Simulaci\'on of the detector: \MAGIC}
\label{chapter:simmagic}

Para conocer de antemano, antes de su construcci'on, la posible
respuesta de nuestro sistema (el telescopio \Cherenkov) a los datos de
entrada (la luz producida en las cascadas atmosf'ericas tras la
llegada de rayos c'osmicos y gamma a la Tierra), necesitamos realizar
una simulaci'on del detector. En nuestro caso se han escrito varios
peque~nos c'odigos de simulaci'on. 

El flujo general del proceso de simulaci'on viene esquematizado en la
fig. \ref{fig:simprocess}. Como ya se dijo, la simulaci'on del
detector se ha dividido en dos etapas:
%
\begin{enumerate}
\item Paso de la luz \Cherenkov generada en las cascadas por la
  simulaci'on de la atm'osfera, y su reflexi'on en el sistema
  multi-espejo de \MAGIC
  
\item Proceso de \emph{pixelizado}, simulaci'on de la l'ogica de
  disparo y el an'alisis de las im'agenes
\end{enumerate}
%
En este cap'itulo se explican con cierto detalle estas dos etapas, y
se presentar'an sus primeros y m'as directos resultados.

%------------------------------------------------------------
\section{Reflexi\'on de la luz \Cherenkov y su llegada al plano focal}
\label{sec:reflcoll}

Si observamos el resultado de \CORSIKA, observamos que tenemos
ficheros de datos con informaci'on acerca de las particulas y de la
luz \Cherenkov que alcanzaron el nivel de observaci'on. Nosotros
utilizamos 'unicamente un nivel de observaci'on, 2\,200\u{m} sobre el
nivel del mar, y estamos interesados solamente en la luz
\Cherenkov. Por tanto, los programas de an'alisis se concentran en
el estudio de esta luz.

Ya vimos en la tabla \fullref{tab:cphstruct} los datos que obten'iamos
para cada fot'on \Cherenkov. Consist'ian, para cada fot'on, de una
7-tupla $\{\lambda,x,y,u,v,t,h\}$, donde:
%
\begin{center}
\begin{tabular}{cl}
$\lambda$ & Longitud de onda del fot'on \Cherenkov emitido \\
$x,y$     & Posici'on del foto'on \Cherenkov en el plano horizontal \\
          & del nivel de observaci'on\\
$u,v$     & Cosenos directores, en los ejes X e Y, de la trayectoria \\
          & del fot'on\\
$t$       & Tiempo transcurrido desde la primera interacci'on hasta \\
          & que el fot'on fue emitido \\
$h$       & Altura de emisi'on del fot'on\\
\end{tabular}
\end{center}

Las valores de $x, y, u$ y $v$ se miden en el sistema de coordenadas
del nivel de observaci'on (denominaremos este sistema como
\emph{sistema del observador}). Sin embargo, usamos otros dos sistemas
en nuestro programa de simulaci'on. En total, utilizamos los
siguientes sistemas de coordenadas (v'ease fig. \ref{fig:coordsys}):
%
\begin{Ventry}{$S" \equiv O"X"Y"Z"$:}
\item[$S \equiv OXYZ$] Sistema de coordenadas del observador. Todas
  las variables de salida de CORSIKA son medidas en este sistema
  
\item[$S^\prime \equiv O^\prime X^\prime Y^\prime Z^\prime$] Sistema del
  multi-espejo del telescopio (asumimos que $O\equiv O^\prime$)
  
\item[$S" \equiv O"X"Y"Z"$] Sistema local de cada espejo individual,
  donde cada fot'on incide
\end{Ventry}

\simprocessfig
%
Para la simulaci'on de la reflexi'on, seguimos el siguiente algoritmo:
%
\begin{enumerate}[1.]

\item Leer par'ametros de la simulaci'on, reservar memoria, 
  inicializar variables y abrir ficheros de datos.

\item Seguir los siguientes pasos para cada fot'on \Cherenkov de cada
  cascada analizada:
  
  \begin{enumerate}[\theenumi.1.]
  \item Calcular, para ese fot'on, la transmisi'on atmosf'erica, y
    decidir si el fot'on sobrevivi'o a la absorci'on.
    
  \item Aplicar la reflectividad de los espejos, posibles zonas
    muertas, simulando sus imperfecciones.

  \item Si el fot'on sobrevive a estos efectos, trazar su trayectoria
    reflejada hacia el plano focal, y guardar esta informaci'on en el
    fichero de salida.
  \end{enumerate}

\item Volver a 2.1 mientras queden fotones o cascadas.

\end{enumerate}

Expliquemos brevemente los pasos m'as importantes.

%%------------------------------------------------------------
\subsection{Simulaci\'on de la absorci\'on atmosf\'erica}

Para simular la atenuaci'on atmosf'erica sufrida por la luz
\Cherenkov, emitida 'esta durante el desarrollo de una cascada,
necesitamos primero estimas de manera realista las variables
participantes en el proceso. 'Estas variables son la \emph{altura
  verdadera en la vertical}, definida, como su nombre indica, como la
altura en la vertical de cada punto de la atm'osfera, y la \emph{masa
  de aire}, definida como la densidad de aire atravesada por los
fotones en su trayectoria. Dado que el \CORSIKA cl'asico, en
particular la versi'on utilizada para este trabajo, proporcional
solamente la altura de emisi'on de un fot'on con respecto del plano
del observador, nos vemos forzados a calcular nuestras variables de
inter'es en los programas de simulaci'on. 'Estas ser'an de hecho de
vital importancia en el caso de estudios a alto 'angulo cenital.

%\coordsysfig
\Systemsfig %% in landscape

%%------------------------------------------------------------
\subsubsection{C\'alculo de la altura verdadera en la vertical}

Podemos seguir usando nuestra versi'on (modificada) de \CORSIKA, la
cual utiliza una atm'osfera plano-paralela, simplemente usando las
siguientes aproximaciones.

Asumiremos que un observador se encuentra en un punto A (v'ease la
fig. \ref{fig:geomview}), a una altura \ho sobre el nivel del mar.
Nuestro observador observa una fuente (que emite rayos gamma), y lo
hace bajo un 'angulo cenital $\theta$. Uno de los rayos gamma de la
fuente acaba de alcanzar la atm'osfera terrestre, y desarrolla una
cascada atmosf'erica. (Para mayor claridad, simplificaremos la
situaci'on y supondremos que la trayectoria del fot'on, y por tanto el
eje de la cascada, intersecta el nivel de observaci'on en el punto A.)

Usaremos la siguiente notaci'on (de la fig.\ref{fig:geomview}):

\begin{tabular}{ll}
A & Punto donde se encuentra situado el observador\\
O & Punto a nivel del mar, en la vertical del observador \\
P & Punto de emisi'on de un fot'on \\
\ho & Altura del nivel de observaci'on \\
\hc & Altura dada por CORSIKA, en la vertical del observador\\
\hv & Altura en la vertical de un fot'on\\
$R \equiv R_\oplus$ & Radio de la Tierra\\
$\theta$ & 'Angulo cenital en el nivel de observaci'on\\
$\omega$ & 'Angulo cenital a nivel del mar\\
$\alpha$ & 'Angulo desde el centro de la Tierra, subtendido entre A y P\\
\end{tabular}

%\geomviewfig

Si $\ho \neq 0$ ni despreciable respecto de \hc, y $\theta \simeq
\omega$, la expresi'on que nos da el valor de la \emph{altura
  verdadera en la vertical} en el punto P, \hv, es:
%
\hveq
%
En el caso de que $\ho \simeq 0$ o despreciable respecto de \hc,
podemos escribir:
%
\hvapproxeq

%------------------------------------------------------------
\subsubsection{Estimaci\'on puramente geom\'etrica de la masa de aire}

En la aproximaci'on mas simple, el factor de masa de aire (AM, la
relaci'on entre el camino 'optico recorrido a un determinado 'angulo
cenital y el camino 'optico recorrido en la vertical) puede ser
estimado 'unicamente mediante consideraciones geom'etricas. En esta
aproximaci'on, el factor de masa de aire resulta:
%
\mgeomeq

%------------------------------------------------------------
\subsubsection{C\'alculo de la masa de aire mediante un modelo de
  densidades exponencial}

La expresi'on dada en \eqref{eq:mgeom} resulta de argumentos
'unicamente geom'etricos. Resultar'ia mucho m'as satisfactorio ir un
paso m'as all'a, mediante la introducci'on de un modelo de densidades
exponencial. En nuestras simulaciones se ha usado un modelo
atmosf'erico representado por 4 exponenciales y una funci'on lineal en
la capa superior de la atm'osfera. Por tanto, aunque el uso de una
'unica exponencial no nos proporcionar'a un resultado del todo
correcto, resultar'a m'as cercano a la realidad.

Utilizaremos el siguiente modelo:
%
\denseq
%
donde $h$ es la altura sobre el nivel del mar, y \Hs es la
\emph{escala de alturas} de la atm'osfera. En la Tierra 'esta resulta
ser aproximadamente $\Hs =
7.4\u{km}$. Con este modelo, el camino 'optico (ignorando la refracci'on)
puede escribirse (con la aproximaci'on $h/R \simeq 0$) como:
%
\optpathredeq
%
N'otese que hemos escrito expl'icitamente que deseamos calcular este
camino para un 'angulo cenital $\theta$, y para unas alturas en la
vertical entre \ho and \hv. Nuestra \emph{masa de aire} $\mathcal{AM}$
vendr'a dada por\footnote{La \emph{masa de aire} en Astronom'ia
  'Optica viene definida por la relaci'on $\mathit{Airmass} \equiv
  {I(\theta;0,\infty)}/{I(0\deg;0,\infty)} $}:
%
\AMdefeq
%
Por tanto, integrando tenemos:
%
\AMfulleq
%
donde $\mathrm{erfc}$ es la \emph{funci'on complementaria de error}.
Una aproximaci'on para bajo 'angulo cenital, o m'as exactamente para
valores de $X\equiv\sqrt{{R\,\cos^2\theta}/{2\,\Hs}}$ relativamente
grandes, es:
%
\AMapproxeq

\paragraph{Correcci\'on por refracci\'on.} Hemos definido $R \equiv
R_\oplus$, el radio de la Tierra. Una sencilla forma de corregir por
refracci'on ser'ia tomar
%
\refracapproxeq

%------------------------------------------------------------
\subsubsection{Comparaci\'on de las diferentes aproximaciones}

Deseamos comparar la expresi'on obtenida para el c'alculo de \hv con
su la aproximaci'on en el caso de que \ho sea suficientemente
peque~no, $h_{\mathrm{v,approx}}$.  Para ello evaluamos la expresi'on
%
\diffhveq
%
para $\ho=2.2\u{km}$ (altura de la futura localizaci'on de \MAGIC),
para diferentes 'angulos cenitales y alturas originales sobre la
horizontal, \hc (v'ease la fig. \ref{fig:hvdiffs}). Podemos ver que la
diferencia aumenta con el 'angulo cenital, tal y como esper'abamos,
aunque 'esta permanece por debajo del 5\% hasta los 85\deg.

\hvdiffsfig
%
Lo siguiente que queremos comparar son las expresiones geom'etricas
para la masa de aire, $m$ y $m_{\mathrm{simple}}$ (aproximaci'on
sencilla que usa $h_{\mathrm{approx}}$), con el valor de
$\mathcal{AM}$, obtenido mediante el uso del model de densidades
exponencial. Incluiremos en la 'ultima expresi'on la correcci'on por
refracci'on propuesta previamente en \eqref{eq:refrac}. En la fig.
\ref{fig:AMcomp} se muestra dicha comparaci'on, para tres valores de
\ho (0\u{km}, 4\u{km} and 2.2\u{km}).  En cada gr'afica se han
utilizado dos valores para la altura vertical, $\hv=5\u{km}$ y
$\hv=100\u{km}$. (La funci'on $\sec(x)$ se muestra por
completitud.)

\AMcompfig

\afterpage{\clearpage}

%------------------------------------------------------------
\subsubsection{C\'alculo de la transmisi\'on atmosf\'erica}

Una vez que hemos determinado la \emph{altura vertical verdadera} de
producci'on de un fot'on, \hv, y su correspondiente \emph{masa de
  aire} $\mathcal{AM}$, solo nos resta utilizar estas variables en el
c'alculo de la transmisi\'on atmosf\'erica. Para ello se han tenido en
cuenta los siguientes tres efectos::

\paragraph{Dispersi\'on Rayleigh.} Las mol'eculas de la atm'osfera
producen una dispersi'on de los fotones. Su efecto est'a fuertemente
correlacionado con la longitud de onda $\lambda$ del fot'on, siendo
m'as importante a longitudes de onda cortas. El coeficiente de
transmisi'on debido a dispersi'on Rayleigh es:
%
\Rayleigheq
%
donde $\tau_{i=1,2} = \tau_0 \, \exp(-h_i/H_{\mathrm{S}}) \sec\theta$
es el \emph{espesor oblicuo} por encima de una altura $h_i$, en el
caso de un perfil exponencial de densidades (en realidad en la
pr'actica hemos usado los valores dados para el modelo atmosf'erico de
CORSIKA), $\tau_{\mathrm{R}}(\lambda\!=\!400\u{nm}) = 2970\u{g/cm^2}$
es el recorrido libre medio de la dispersi'on Rayleigh en t'erminos
del espesor, $\tau_0 = 0.00129\u{g/cm^2}$, y $H_{\mathrm{S}}$ es la ya
mencionada escala de alturas de la atm'osfera. En los programas se
hayan precalculados estos coeficientes para diferentes longitudes de
onda.
  
\paragraph{Dispersi\'on Mie.} Se debe a particulas aerosoles (polvo)
suspendidas en el aire, siendo su efecto mayor a menores
alturas. Depende en general del tama~no de las part'iculas. Para un
modelo como el nuestro, el coeficiente de transmisi'on es:
%
\Mieeq
%
donde $l_{\mathrm{M}}(\lambda\!=\!400\u{nm}) \simeq 14\u{km}$ es el
recorrido libre medio de dispersi'on Mie, y $h_{\mathrm{M}} \simeq
1.2\u{km}$ es la escala de alturas de la distribuci'on de
aerosoles. Tambi'en hemos precalculado su valor para diferentes
longitudes de onda, aunque esta dependencia es peque~na.
  
\paragraph{Absorci\'on por Ozono.} Este efecto es muy importante en el
rango de 280--340\u{nm} y para cascadas de baja energ'ia, para las
cuales la mayor'ia de la emisi'on \Cherenkov se localiza en capas
altas, donde la concentraci'on de Ozono es mayor (20--30\u{km}). Para
su estimaci'on nos valemos de la extinci'on en magnitudes por unidad
de masa de aire:
%
\amagneq
%
donde $k(\lambda)$ es el coeficiente de absorci'on en \u{cm^{-1}}, y
$T(\hv)$ es la concentraci'on total de Ozono para una regi'on t'ipica
tropical (elegimos este valor ya que se aproximaba bastante bien al
perfil real de La Palma), sobre la vertical en cada posici'on, en
\u{atm cm}. El coeficiente $1.11$ se eligi'o por corresponder con el
espesor 'optico para el Ozono a 320\u{nm}, seg'un \cite{Elterman:book}.

Con este valor, podemos estimar la extinci'on en magnitudes por masa
de aire entre dos alturas, y usando la ley de Pogson podemos
determinar el coeficiente de transmisi'on, que resulta ser:
%
\Ozoneeq

En consecuencia, el coeficiente total de transmisi'on en la atm'osfera
ser'a:
%
\TotalTransmissioneq

%%------------------------------------------------------------
\subsection{Reflexi\'on en los espejos}

Para un fot'on que haya sobrevivido a la absorci'on atmosf'erica,
aplicaremos la reflectividad de los espejos, as'i como otros efectos
menores, y calcularemos la trayectoria reflejada, y el punto de
incidencia de 'esta con el plano de la c'amara. En nuestra simulaci'on
hemos supuesto una reflectividad constante en el intervalo de
longitudes de onda de intere's, $[290\u{nm},600\u{nm}]$.

En general, la direcci'on de llegada de un fot'on no coincidir'a con
la direcci'on a la que apunta el telescopio (y si aquella se encuentra
fuera del campo visual del telescopio, el fot'on, de ser reflejado, no
caer'a en la c'amara). Asociamos un sistema de coordenadas al
telescopio (v'ease la fig. \ref{fig:coordsys}), $S'$. El sistema en el
nivel de observaci'on es $S$, y es en 'este sistema donde tenemos los
par'ametros de la trayectoria del fot'on, el vector direcci'on,
$\mathbf{r}$, y el punto de intersecci'on de la trayectoria con el
nivel de observaci'on, $\mathbf{x}$. Debemos por tanto hacer un cambio
de sistema de coordenadas:
%
\rotateAeq
%
donde \thetaCT y \phiCT son los 'angulos que definen la direcci'on a
la que apunta el telescopio, y $\Omega(\theta,\phi)$ viene definida como:
%
\omegaAeq
%
Aqu'i $R_x(\alpha)$ es una rotaci'on de 'angulo positivo del espacio
en torno al eje $X$ de $\alpha$ grados (id. para $Y$ y $Z$). Por
tanto, la expresi'on de la matriz de cambio de sistema de coordenadas es:
%
\transformAeq
%
con $\theta=\thetaCT$ y $\phi=\phiCT$. Una vez que nos hayamos en el
sistema $S'$, buscamos en qu'e espejo individual incidi'o el
fot'on. En nuestra simulaci'on tenemos 920 espejos individuales de
$50\times 50\u{cm^2}$, localizados en posiciones realistas, con la
inclinaci'on adecuada para el enfoque de la luz en la c'amara. Existen
de hecho peque~nos espacios vacios entre espejos adyacentes, y un
fot'on podr'ia ``perderse'' por uno de estos agujeros.

\collimationfig
%
Si un fot'on en efecto incide en un espejo, calcularemos la
trayectoria reflejada a partir del punto de incidencia en el espejo, 
$\mathbf{x}_{\mathrm{cut}}$. Para ello, es conveniente volver a
cambiar de sistema de coordenadas, desde el del telescopio hasta uno
local al propio espejo. Esto significa ahora no s'olo una rotaci'on,
sino tambien una traslaci'on del origen hasta el centro del espejo. La
transformaci'on resultante es:
%
\transformBeq
%
donde $\mathbf{x}_{\mathrm{m}}$ y $(\thetam,\phim)$ dan la posici'on
del centro del espejo en el sistema del telescopio $S'$, y la
direcci'on del vector normal al espejo en su centro, en el mismo
sistema.

Con estos nuevos $\mathbf{x}"$ y $\mathbf{r}"$, calculamos la
trayectoria reflejada, dada por el nuevo vector
$\hat{\mathbf{r}}"$. Finalmente, utilizamos 'esta para obtener la
posici'on final del fot'on en el plano de la c'amara.

\subsection{Focalizaci\'on de los espejos}

Todos y cada uno de los espejos de \MAGIC se encuentra apuntando hacia
el eje 'optico del telescopio, con un 'angulo y una distancia focal
determinada por dos factores.

\begin{enumerate}[i.]
\item La posici'on del espejo en el plato del sistema, y
\item La focal deseada para el sistema global
\end{enumerate}

Este 'ultimo par'ametro, la \emph{distancia focal global}, se ha
fijado en $F=1700\u{cm}$.

Asumamos que enfocamos cada espejo \emph{al infinito}.
El grueso de nuestros datos, sin embargo, son sucesos cuyo m'aximo
desarrollo de la cascada se alcanza aproximadamente a $12\u{km}$ sobre
el nivel del mar, es decir, una a altura de $\hv\simeq 10\u{km}$ sobre
nuestro nivel de observaci'on.  Por tanto, para que nuestro sistema se
encuentre optimizado a esta altura, deberemos \emph{desenfocarlo}, es
decir, mover la c'amara ligeramente. Este efecto se ha dado en llamar
\emph{colimaci'on}. En la fig. \ref{fig:collimation} pod'emos ver un
esquema que nos ayuda a comprender este comportamiento.  

Por simplicidad, asumamos que el telescopio se encuentra apuntando al
cenit. Un espejo individual se encuentra a una distancia $r$ del eje
'optico del telescopio $\overline{\mathrm{OP}}$. Este espejo se
encuentra fijado a un plato de forma parab'olica, y por tanto a una
altura respecto del centro del plato (y que definiremos como dentro
del telescopio) $\mathrm{O}$ de
%
\zparabeq
%
con $F\equiv\overline{\mathrm{OF}}$ la mencionada \emph{distancia
  focal global}. La recta $\mathbf{n}\equiv\overline{\mathrm{PQ}}$ es
la perpendicular al espejo por su centro, siendo $\mathrm{P}$ la
intersecci'on entre esta recta y el eje 'optico del telescopio. La
l'inea $\mathbf{r}$ muestra la trayectoria de un fot'on proveniente
del infinito. El fot'on, una vez reflejado, cortar'a el eje 'optico en
$\mathrm{F}$, por definici'on. Pero, como ya se dijo, el grueso de los
fotones provendr'an de alturas en torno a $\hv\simeq 10\u{km}$ sobre
el telescopio, es decir, la trayectoria de un fot'on que incida en el
centro del espejo ser'a $\mathbf{s}$. El 'angulo entre 'esta y la
trayectoria ``original'' es $\widehat{\mathbf{rs}}\equiv\zeta$. El
fot'on reflejado cruzar'a el eje 'optico esta vez en $\mathrm{C}$.

Por tanto, para optimizar nuestro sistema a 10\u{km} sobre el nivel de
observaci'on, deberemos desplazar nuestra c'amara desde $\mathrm{F}$ hasta
$\mathrm{C}$. De la fig. \ref{fig:collimation} obtenemos las relaciones:
%
\relationsAeq
%
y
%
\relationsBeq
%
Por tanto, el desplazamiento que debemos aplicar viene dado por:
%
\collimationeq
%
En nuestro caso, este desplazamiento es del orden de
2.9--3.3\u{cm}$\approx$3\u{cm} (v'ease fig. \ref{fig:colldelta}),
dependiendo de la posici'on del espejo. Diremos que esta nueva
posici'on de la c'amara es la \emph{posici'on de enfoque 'optimo}.

Sin embargo, podr'iamos haber enfocado los espejos desde un principio
a esta altura de 10\u{km}. Haciendo esto, evitamos cualquier
desplazamiento de la c'amara desde su posici'on nominal
$\mathrm{F}$. N'otese que, en cualquier caso, \emph{las estrellas de
  nuestro campo visual estar'an desenfocadas}, y por tanto ofrecer'an
im'agenes de un tama~no mayor que en el caso de tener el telescopio
enfocado al infinito.

\colldeltafig

En las figuras \ref{fig:hlfivekm}.a a \ref{fig:hlinf}.a mostramos la
imagen de una fuente de luz puntual, situada a varias alturas \hv en
la vertical del telescopio (el cual apunta al cenit), en el caso en el
que el telescopio se encuentra enfocado \emph{al infinito}, y para
diferentes posiciones del plano de la c'amara. Podemos ver que cuando
la fuente se encuentra en el infinito y el plano de la c'amarea es el
plano focal, obtenemos la imagen de menor tama~no. En el caso en que
la fuente se encuentra a $\hv=10\u{km}$, vemos que la posic'on 'optima
se obtiene ahora para una distancia del plano de la c'amara de
1703\u{cm}, como hab'iamos calculado. El hex'agono verde representa el
p'ixel central de la c'amara. La reflectividad utilizada es del 100\%,
se ha realizado una reflexi'on matematicamente exacta, y se ha
eliminado cualquier efecto atmosf'erico.

%\hfinfhlfivekmfig

%\hftenkmhlfivekmfig

%\hfinfhltenkmfig

%\hftenkmhltenkmfig

%\hfinfhltwelvekmfig

%\hftenkmhltwelvekmfig

%\hfinfhlhundredkmfig

%\hftenkmhlhundredkmfig

%\hfinfhlinffig

%\hftenkmhlinffig

\hlfivekmfig
 
\hltenkmfig
 
\hltwelvekmfig
 
\hlhundredkmfig
 
\hlinffig

\Mspotsfig

\afterpage{\clearpage}

En las figuras \ref{fig:hlfivekm}.b a \ref{fig:hlinf}.b observamos
algo similar, pero ahora cno el sistema directamente enfocado a una
altura de 10\u{km}. Es, por tanto, al colocar la fuente de luz a esta
altura y el plano de la c'amara en su posici'on nominal de 100\u{cm},
cuando obtenemos la imagen m'as peque~na posible. En estas figuras la
escala de colores se ha normalizado la la m'axima concentraci'on de
luz (fuente de luz en el infinito y plano de la c'amara a 1703\u{cm},
para los casos a., y fuente de luz a 10\u{km} y plano de la c'amara a
1700\u{cm}, para los casos b.).

En la fig. \ref{fig:Mspots} observamos la imagen de una \texttt{M},
rodeada por un cuadrado, hipot'eticamente generada por un proyector
situado a nuestra posici'on de referencia de 10\u{km}, apuntando hacia
abajo, y con un tama~no en el suelo comparable al del multi-espejo de
\MAGIC. De nuevo, la imagen m'as peque~na se obtiene para una
posici'on del plano de la c'amara de 1700\u{cm}. Observamos como la
imagen se invierte cuando el plano de proyecci'on se mueve de un lado
al otro del plano focal.

Otro efecto que se ha de tener en cuenta es el siguiente: la
\emph{superficie focal} real deja de ser un plano. En general,
consistir'a en una superficie de revoluci'on, similar a un
paraboloide. Para un espejo esf'erico de radio de curvatura
$R_{\mathrm{c}}$, esta superficie se situa a una distancia
$R_{\mathrm{c}}/2$ del centro del espejo, para un haz de luz incidente
paralelo al eje 'optico ('angulo de incidencia de 0\deg).  Si el
'angulo de incidencia es diferente de 0\deg, la mancha de luz de menor
tama~no se sit'ua a distancias algo menores de $R_{\mathrm{c}}/2$.
Este efecto es el llamado \emph{acortamiento de la distancia focal}, y
su magnitud depende del 'angulo de incidendia de la luz.

En el caso de un espejo teselado como el de \MAGIC, y un haz de luz
incidente paralelo como el descrito a 0\deg respecto del eje 'optico
del sistema, todos los espejos individuales recibir'an la luz a
'angulos distintos de 0\deg. En particular, la distancia para la cual
para cada espejo obtenemos la mancha de luz del menor tama~no posible
decrecer'a seg'un aumentamos la distancia al eje del sistema. Para
conseguir que todos los espejos superpongan estas menores manchas
posibles en un solo punto, el plano focal, debemos ir
\emph{aumentando} la distancia focal de los elementos individuales. 

Teniendo esto en mente, todos los espejos han sido agrupados en siete
clases o \emph{zonas}, caracterizada cada una de ellas por una
distancia focal fija. Estas zonas dividen el plato de \MAGIC en ocho
pseudo-anillos concentricos (v'ease fig. \ref{fig:zones}). Como ya
dijimos, las distancias focales de estas zonas $f_i$ deben ser mayores
que $F$, con el objeto de cumplir con el hecho de poseer una prefijada
distancia focal del sistema. Los valores obtenidos para estas
distancias focales se muestran en la tabla \ref{tbl:zonefocals}.

\zonefocalstbl

\thetaspotsfig

\afterpage{\clearpage}

N'otese que hemos usado en las figuras \ref{fig:hlfivekm} a
\ref{fig:Mspots} un sistema \emph{perfecto}: reflectividad 100\%, sin
imperfecciones ni absorci'on atmosf'erica. Si aplicamos una
reflectividad menor, posibles desviaciones de los ejes en el ajuste de
los espejos, imperfecciones en sus superficies, y el efecto de la
atm'osfera, obtendremos una variaci'on de los resultados como la
mostrada en la fig.  \ref{fig:spotsh}.

%%------------------------------------------------------------
\subsection{Plate-scale in the camera of \MAGIC}

We are now interested on the \emph{plate-scale}, that is, the
equivalence between linear distance in the camera plane, and angular
distance in the sky. For this, we change the Zenith Angle of our
point-like source (initially 0\deg) to several values. In
Fig.\ref{fig:thetaspots} we can see the results. Just for completion, we
show the effect of the height of the source, although we are only
interested in the height $\hv=10\u{km}$ above the detector --- the one
used for the optimal-focus position.

The value taken for the plate-scale in the camera plane of \MAGIC is:
%
\platescaleeq

%%------------------------------------------------------------
\section{Detection in the camera}

At the end of the first stage in the detector simulation process, we
will have, for each shower, a bunch of photons in the camera plane.
The next stage will be the so called \emph{pixelization}.  This
process, and others, is performed by the program \texttt{camera}. In
few words, the steps taken in this program are:

\begin{enumerate}
\item Pixelization. Each photon is assigned to a photomultiplier

\item Inclusion of Light of the Night Sky (LONS)
  
\item Simulation of Quantum Efficiency and other effects (light
  guides, plexiglas --- if any --- and first dynode collection
  efficiency)

\item Simple simulation of electronic chain

\item Trigger logic simulation

\item Characterization of images in the camera
\end{enumerate}

\zonesfig

\zonesfocalsfig

%%------------------------------------------------------------
\subsection{Pixelization}

In Fig.\ref{fig:triaxis} we can see the definition of the tri-axial
frame used for the numbering of the pixels. The pixels will have an
hexagonal shape. If the edge-to-edge width of a pixel is $w$ and the
corner-to-corner width is $\hat{w}=(2/\sqrt{3}) w$, then the unit in
each of the three axes is $1.5\times (\hat{w}/2)$ (in the
implementation we take an scaling factor such that $w=1.0$).  With
this definition the coordinates of the pixels are, for example for the
first seven pixels\footnote{We are using the so called
  \emph{spiral-numbering}, where the first pixel is the one in the
  center, and then for each ring, starting from the innermost one, we
  give sequential numbers in counterclockwise order.} shown in the
Fig.\ref{fig:triaxis}:
%
\begin{center}
  \begin{tabular}{rc}
    1 :& ( 0, 0, 0) \\
    2 :& ( 1,-1, 0) \\
    3 :& ( 1, 0,-1) \\
    4 :& ( 0, 1,-1) \\
    5 :& (-1, 1, 0) \\
    6 :& (-1, 0, 1) \\
    7 :& ( 0,-1, 1) \\
  \end{tabular}
\end{center}

The primary question can be expressed in the following terms:
\emph{What is the pixel number $k$ of the PMT where a photon falls
  in?}.  Following \cite{Fu:hexgrid}, we get an elegant solution to
the problem. Indeed, what we think of as an uniform hexagonal grid,
can be isometric projection\footnote{An isometric projection is an
  orthographic, i.e., non-perspective, projection onto the $x+y+z=0$
  plane.} of an infinite grid of unit cubes whose centers satisfy the
equation
%
\planeeq
%
In this particular case, the problem of determining which hex contains
a given point becomes the problem of which cube contains a point.
Additionaly, only the transformations from the plane to the cube grid
and vice versa is needed.

With this approach, the the formula that gives the answer to our
question is:
%
\hexgrideq
%
where $\mathbf{r}$ is your position, $r_i$ is the $i$-th component of
the vector $\mathbf{r}$ and $\mathcal{R}$ is a rounding function.

\axesfigs

%\triaxisfig

With this definition of the axes, we pre-calculate the coordinates
$(x^\prime,y^\prime,z^\prime)$, in order to use them afterwards in the simulation code.
Of course, only two of these coordinates are independent, since we
have the constraint \eqref{eq:plane0}\footnote{Note that the algorithm
  is symmetric in $X^\prime$, $Y^\prime$ and $Z^\prime$, since the axes are symmetric.}
  
Now, two of the coordinates (only two are independent), say
$\{x^\prime,y^\prime\}$, are used to identify each pixel in an array $A$, which
has in its cells the number of pixel to which those coordinates
belong.

This algorithm, however, does not have {\itshape a priori} information
about the position of the center of the pixels, information that is
indeed needed in the simulation code. So, we have to calculate in
addition the coordinates of the centers of the pixels. For this
purpose, yet another different coordinate system is used, the one
shown in Fig.\ref{fig:biaxis}. The units of the axes X and Y are
millimeters, and the units of the axes I and J are both two times the
apotheme of one pixel ($2 \times a$).

%\biaxisfig

With this definition of the axis, we use the intermediate
$(i,j)$-coordinates, in order to use them to calculate the
$(x,y)$-coordinates of the centers. To get these $(x,y)$-coordinates,
we just apply the following change os system:
%
\bitoeucleq

%%------------------------------------------------------------
\subsection{Contribution of the Light of the Night Sky}

The amount of diffuse Light of the Night Sky (LONS) in La Palma, at
the site where \MAGIC will be locate, has been measured
\cite{Razmick:nsb}, leading to a flux 
%
\LONSeq
%
Let's estimate the number of photons that the LONS contribute to each
pixel in the camera of \MAGIC. For this calculation, the following
parameters have to be taken into account:
%
\begin{center}
\begin{tabular}{lrl}
Mirror surface & $S_{\text{mirror}}$ &$= 230 \u{m}^2 $ \\
Reflectivity & $R$ &$= 85\% $ \\
Light guides efficiency & $\epsilon_{\text{l.guides}}$ &$= 90\% $ \\
Transmittance of PMT window & $\epsilon_{\text{window}}$ &$= 95\% $ \\
First dynode collection efficiency &
             $\epsilon_{1^{\mathrm{st}}\text{dyn.coll.}}$ &$= 90\%$ \\
Pixel angular size & $\theta_{\text{1pixel}}$ &$= 0.1^\circ$ \\
Mean \QE folded with LONS & \QElons &$\sim 13\% $ \\
\end{tabular}
\end{center}
%
With this, we have that the half angular size of a pixel is
$\theta_{\frac{1}{2}\text{pixel}} = 0.05^\circ$, and the solid angle
for each pixel $\Delta\Omega =
2\pi(1-\cos\theta_{\frac{1}{2}\text{pixel}}) = 2.4\cdot 10^{-6}
\u{sr}$. Then, the \emph{mean number of photons} arriving at the
entrance of the pixel in $1\u{ns}$ is:
%
\Nineq
%
and using the mean \QE for the LONS, \QElons:
%
\Ninbiseq
%
As an example, if we use then a gate of $\Delta T=5\u{ns}$, we arrive
at a \emph{mean contribution of LONS per pixel per gate} of:
%
\LONStimeeq

%%------------------------------------------------------------
\subsection{Simulation of the response of a PMT}

The simulation of the response of a PMT, despite of the complexity of the
process, depends mainly of a well known set of effects, namely:

\begin{enumerate}[a.]
\item the Quantum Efficiency (\QE) of the PMT, function of the
  wavelength of the incident photon,
  
\item the natural fluctuations of this QE for any given fixed
  wavelength,

\item the first dynode response, 

\item the possible afterpulsing, 

\item the single photoelectron response of the PMT
\end{enumerate}

Let's study the process of conversion of normal photons (\Cherenkov
photons, photons comming from the LONS, and starlight) into number of
photoelectrons after the photocathode of the PMT.

%------------------------------------------------------------
\subsubsection{Fluctuations taking place}

Before we try to understand this process of conversion of \Cherenkov
photons into photoelectrons, let's try to identify the possible
sources of fluctuations.

In the first place, we have a given number of incoming \Cherenkov
photons. This number fluctuates from shower to shower, due to the
statistical nature of the generation of the atmospheric shower: even
for a fixed energy of the primary particle which generated the shower,
the probabilistic generation of secondary particles, the fluctuations
in the height of the first interaction (and hence in the height where
the maximum particle generation is achieved) and the random generation
of \Cherenkov photons by charged particles will lead to a fluctuating
number of Cherenkov photons.

Let's call \Nphot the \emph{input number of photons in the
  photomultiplier tube}. Now we want to simulate a measurement of
\Nphot leading to a certain current $I$, or a given amount of
electrons in the anode, $N_{\mathrm{e}}$. The \Cherenkov photons
hitting the photocathode will produce a given number of
photoelectrons. This process is probabilistic, and depends on the
Quantum Efficiency, \QE, of the photocathode.  But this \QE depends
not only on the wavelength $\lambda$ of the incident photon, but also
on the specific place of the photocathode where the photon hits.
Additionally, the measured \QE (for a given $\lambda$) of a PMT is
just an average value over many photons (of this $\lambda$). This
means that the \QE itself must be seen as a fluctuating term, in the
sense of its probabilistic nature.

After the photoelectrons are emitted from the photocathode, a certain
number of them will arrive at the first dynode. Of course, this
depends on the place where the photoelectron comes from, and on the
design of the PMT itself. This dependency is given by the so called
\emph{first dynode collection efficiency}. Different measurements lead
to a value of about 90\%. After hitting the first dynode, each
photoelectron can liberate a typical number of 6 electrons, but again
this can fluctuate.

We just started the cascading process. After the first dynode comes
the second, where we again have a multiplicating term (and hence an
efficiency term). This multiplicating process continues until we reach
the anode, where we finally have a big number of electrons. This
number depends on the gain and the voltage of our PMT.

This cascading process can be simulated by using the so called
\emph{single photoelectron response} or \emph{single electron
  spectrum} (SES) of the photomultiplier, which is nothing but the
distribution of the output that we get from the photomultiplier for a
single photoelectron release by the photocathode. This depends also on
the high voltage applied to the PMT. Therefore, by using this
distribution, we can simulate the output of each single photoelectron,
superpose all these output signals, and we will get at the end a
realistic view of the response of our PMT to the incident amount of
light.

%------------------------------------------------------------
\subsubsection{Simulation of the Quantum Efficiency}

We have assumed that \emph{a single photon cannot produce, for
  whatever processes inside the photocathode, more than one
  photoelectron}. Therefore, for each single photon, the generation of
a photoelectron is what is called a \emph{Bernoulli process}. For such
processes, we can have only two outcomes: \emph{success} (with
probability $p$), or \emph{fail} (with probability $1-p$). For a
given, fixed number of incoming photons, the generation of
photoelectrons is a good example of a \emph{binomial} process. Let's
forget first about the dependency of \QE with the wavelength.  Let's
assume, therefore, that we have a monochromatic bunch of photons, and
call call $\QEo=\QE(\lambda_0)$, the Quantum Efficiency (the
``average'' value, obtained by measurements) of the photocathode at a
fixed wavelength $\lambda_0$.

\noutphotfig

We can see that each incoming photon have a certain probability \QEo
of generating a photoelectron. This can be simulated by using a
\emph{uniformly distributed random variable} $X$ in the interval
$[0,1]$ for each photon: whenever the value of $X$ is smaller than
\QEo, we assume the photon did generate a photoelectron; otherwise, it
didn't:
%
\begin{enumerate}
\item Take photon.
\item Generate uniform random number $r$ in $(0,1)$.
\item If $r < \QEo$, take photon, else go to 1.
\item Are there photons left? If yes, go to 1.; else Stop.
\end{enumerate}
%
Of course, after following this algorithm \Nphot times, we will have,
\emph{on average}, a number of photoelectrons given by
%
\nmeaneq
%
The words \emph{on average} mean simply that we will not get, from a
single realization of our experiment, exactly a number \Nmean of
photoelectrons (at least, not always). If we repeat this experiment a
lot of times, we will get instead a distribution of numbers \Ntrial,
corresponding to the number of photoelectrons produced. This
distribution, given the \emph{binomial} nature of the process, will
result in a \emph{binomial distribution}. The mathematical expresion
for this is:
%
\binomeq
%
where $\mathcal{P}_{\mathrm{Binom}}(r)$ is the probability of getting
$r$ successes out of $N$ independent trials, each of them with only
two possible outcomes: success (with probability $p$) or failure (with
probability $q=1-p$). It can be shown that the expectated value of the
number of successes is $\bar{r}=\sum r\mathcal{P}(r) = N p$, which is
far from surprising. For this distribution we have also that
$\sigma^{2} \leq \bar{r}$, where the equality holds only for $p=0$. In
general the variance is smaller than the mean. This is so because the
upper limit imposed on $r$ (which cannot be larger than $N$) reduces
the spread of the $r$-distribution.

\noutqesmallfig

\varquotfig

The main use of the binomial distribution is in the limits:
%
\begin{itemize}
\item $p\rightarrow 0$, $N\rightarrow\infty$, but $Np=\mu$ (constant),
  when Binomial $\longrightarrow$ Poisson, and
\item $p=const.$, $N\rightarrow\infty$, when Binomial
  $\longrightarrow$ Gaussian
\end{itemize}
%
In our case, $N=\Nphot$, $p=\QEo$ and $Np=\Nmean$.  This means that,
\emph{for \Nphot large and in the case of $\QEo\rightarrow 0$} (more
generally speaking, for \QEo small) the distribution of the number of
outcoming photoelectrons \Ntrial will be very similar to the
distribution obtained by using the mean number of photoelectrons
\Nmean as the mean value of a Poisson distribution. More clearly, the
requierements of \Nphot large and \QEo small are needed if we want to
use this procedure, what I will call the \emph{Poisson approach}.

The use of the \emph{Poisson approach} is very attractive because you
don't need to work with separate photons. Moreover, some times it is
the most direct approach: you can get your photons at the entrance of
the PMT in bunches of several tens of photons. In this case it's more
comfortable to use mean values and the corresponding Poisson
distribution: we are sure that $N$ is going to be large enough. But we
still have to be sure that our \QEo is small enough to allow us to use
this approach. Fortunately, this is normally the case.

In order to show the possible deviations we can get by using the
\emph{Poisson approach}, a simple simulation was done: a pre-fixed
number of photons \Nphot was sent to a hypothetic photocathode, where
we simulated the Bernoulli process of the production of a
photoelectron by using the simple algorithm shown above.  The \QEo of
the photocathode was also fixed, and we counted the number of
photoelectrons emerging from it \Ntrial.  In addition, using the
number of incoming photons and \QEo, we estimated the average number
of photoelectrons \Nmean we should get, and then followed the
\emph{Poisson approach} to get the final number of photoelectrons
\Nrand.

To show how we can get severe deviations from the right behavior by
using the \emph{Poisson approach}, I performed the simulation varying
the number of incoming photons and the value of \QEo. The results are
shown in Figs. \ref{fig:distrib1}, \ref{fig:varwithqe}
\ref{fig:varwithp}. In Fig. \ref{fig:distrib1} we see the effect of
using a \QEo very different from $0$. For small values of \QEo, both
distributions of photoelectrons \Ntrial (solid line) and \Nrand
(dashed line) appear reasonably similar. However, as \QEo is diverging
from our hypothesis of \QEo small, both distributions get more and
more different. In Fig. \ref{fig:varwithqe} we can see how different
these distributions are for different \QEo, by using the quantity
%
\varquoteq
%
Although the mean value remains practically the same for both
distributions, no matter the value of \QEo, the variance
$\sigma^2(\Nrand)$ increases much more than the variance
$\sigma^2(\Ntrial)$ with \QEo increasing. As we said before, this is
just the result of the upper limit imposed in \Ntrial (it cannot be
larger than \Nphot).

%------------------------------------------------------------
\subsubsection{Simulation of Single Electron Spectrum}

This is just an application of the general procedure of getting a
series of random numbers following a user-defined distribution, called
the \emph{Acceptance-Rejection Method} (Von Neumann). Let's assume we
know the \emph{single electron spectrum}, $S(x)$, defined in an
interval $(a,b)$. This will be in principle a not normalized function
proportional to the normalized probability density function (p.d.f.)
of distribution we want to simulate. Then, choose a p.d.f. uniform on
the interval $(a,b)$. Find a constant $C$ such that $C$ times this
uniform p.d.f.  is everywhere greater than or equal to $S(x)$. This
scenario is shown in Fig. \ref{fig:vonneumann}.

\vonneumannfig

First, simulate a random value $x$ uniformly on $(a,b)$. Then generate
a $y$ on $(0,C/(b-a))$. The point $(x,y)$ will uniformly populate the
box shown in Fig. \ref{fig:vonneumann}. If $y\le S(x)$, we accept $x$
as then next value f the random number. If $y\geq S(x)$, reject $x$
and try again.  This method is very simple and has an efficiency
(fraction of values $x$ accepted) of
%
\vonneumanneffeq
%
(Note that if the function $S(x)$ has sharp peaks, the efficiency can
be very low; one can then use different constants for different
regions in the interval $(a,b)$.)

\timeresponsefig

For each photoelectron we will get, by using this procedure, the
amplitude of the registered signal. In order to simulate a realistic
signal at the output of our PMT, we should use the arrival time of
each photon. On top of this time, the PMT introduces a delay. This
delay follows a distribution similar to a gaussian. The time between
the arrival of a delta-function light pulse and the time where the
output signal reaches its maximum is called \emph{electron transit
  time} (ETT). The delay between our input light and the output signal
comes then governed by the ETT and the \emph{transit time spread}
(TTS, also called \emph{transit time jitter}, the FWHM of the
distribution of delays). The output signal is characterized by its
\emph{rise time} (time where the output signal rises from 10\% to 90\%
of the maximum amplitude). Sometimes, instead of having the \emph{rise
  time}, $t_{\mathrm{rise}}$ we have the FWHM of the output
characteristic, $\approx 2.36\sigma$. The situation is schematized in
Fig. \ref{fig:timeresponse}. With all this we can reproduce the
response of our PMT to the bunch of incident photons.

After all this chain of events, we can then introduce either the
trigger logic of our system, or any electronic device which modifies
this signal.

%%------------------------------------------------------------
\subsection{Simulation of the signal processing chain}

In Fig.\ref{fig:pixelreadout} is was shown the basic electronic chain
for every single pixel in the camera of \MAGIC. For an incident photon
the realistic signal produced is shown in Fig.\ref{fig:pulse}. After
the cascading process in the PMT, we will have an electronic current
$I=I_{\mathrm{A}}$, equal to the area A shown in this figure. Due to
the \emph{AC coupling}\footnote{The \emph{AC coupling} is the use of a
  special circuit to remove the static (DC) components from the input
  signal to the amplifier in an instrument, leaving only the
  components of the signal that vary with time [??].}
%%  [http://www.measurementsgroup.com/Guide/indexes/g\_index.htm].} 
used in the design of our electronic chain, after a single pulse there
is a charge equal to the one coming from the pulse, which developes in
an inverse current. The result is a positive pulse (positive voltage)
after the incoming pulse dies (this is represented in
Fig.\ref{fig:pixelreadout} by the area B). The fading out of this
``remmanent pulse'' is characterized by the constant $\tRC=RC$ of the
circuit. This value is chosen, in the case of \MAGIC, to be around
$\tRC=5\,\mu\mathrm{s}$.

Since the rate of incoming pulses is typically larger than this \tRC,
the \emph{zero-voltage level} will become a baseline with a voltage
\emph{greater that zero}. The baseline will fluctuate randomly around
a given level: this level depends on the value of \tRC and the rate of
incoming pulses.

In the simulation programs, we have the option of using or not the
simulation of this fluctuating baseline, as well as the discretization
of the pulses in time bins. In case of using this later option, the
signal is integrated in a defined number of bins that depends on the
trigger gate. In case of not using this discretization, the whole
charge developed in the pulse is integrated. I will explain this more
in detail in section \ref{sec:triggerlogic}.

\pulsefig

%%------------------------------------------------------------
\subsection{Trigger Logic}
\label{sec:triggerlogic}
%
First, we defined a \emph{trigger area} in the camera: the area where
we allow the pixels to give trigger is not the whole camera. The
defined \emph{trigger area} is a disc centered in the center of the
camera, with a fixed radius. In our investigations we used different
sizes for this area, and even the whole camera. The estimation of the
optimal radius is as follows. We are interested mainly in the low
energy gamma-ray showers, i.e., with primary energies in the range
$10-100\u{GeV}$. The maximum development of this showers, as we said
above, is at around $10-12\u{km}$ a.s.l., this means a height of
$\hv\simeq 8-10\u{km}$, when looking at the Zenith. On the other hand,
the \Cherenkov pull of light extends up to about $\rhump\simeq
100-120\u{m}$ from the shower axis: let's take $\rhump=110\u{m}$.
This distance fixes an upper bound for the area where we have the
maximum detection efficiency for these showers. Looking from the
telescope, and in the case of looking at the Zenith, this distance
corresponds to an angular distance calculated as
%
\trigradeq
%
The value of $\varphi$ oscillates in the range $\simeq
0.7\deg-0.8\deg$, for primaries in the energy range indicated. We
chose the larger, more conservative value of 
%
\phitriggereq
%
However, we should not make this area too big, since we will not
notice an improvement in the rate of the gamma showers in our energy
range of interest, but we \emph{will} get an increment of the
backaground rate proportional to the square of the increment in radius
in the trigger area. In Fig.\ref{fig:trigarea} we see how big in the
whole camera is this area of $0.8\deg$ radius.

\trigareafig

Inside this \emph{trigger area} is where we define the different
trigger conditions. The most elementary trigger conditions are the so
called \emph{multiplicity triggers}:

\begin{description}
\item[\emph{Multiplicity trigger} \trigM{n}{q_0}:] An image is said
  to give trigger if there are at least $n$ pixels in the
  \emph{trigger area} with a charge of at least $q_0$ photoelectrons
  each.
\end{description}

\noindent
Another kind of trigger conditions are the \emph{next-neighbour
  triggers}:

\begin{description}
\item[\emph{Next-neighbour trigger} \trigNN{n}{q_0}:] An image is said
  to give trigger if there is a simply connected area of at least
  $n$ pixels in the \emph{trigger area} with a charge of at least
  $q_0$ each.
\end{description}

\noindent 
We should note that for fixed values of $n$ and $q_0$ we can define a
subset of \emph{next neighbour trigger} conditions, namely, those
which require that those pixels in the simply connected area, form
what is called a \emph{closed packet}. We will denote this sub-class
of conditions with the symbol \trigNNc{n}{q_0}.

Yet another type of trigger conditions, a superset of the
\emph{next-neighbour} conditions, are the so called \emph{topological
  triggers}. For this type of conditions we must define a
\emph{topology} of pixels $S$, consisting of a set of pixels following
a given geometrical pattern.

\begin{description}
\item[\emph{Topological trigger} \trigTop{S}{q_0}:] An image is
  said to give trigger if there is a pattern of pixels coincident with
  the pattern given by $S$, with a charge of at least $q_0$ each.
  (Actually, a different minimum number of photoelectrons can be
  defined for each pixel in $S$.)
\end{description}

The studies presented in this work have been done using the following
trigger conditions (the notation ``$a:b$'' denotes the range of
integer values from $a$ to $b$):
%%
\begin{center}
  \begin{tabular}[t]{rl}
    \trigNN{k}{q_0} & Next neighbour trigger with multiplicity \\
    & $k=3:5$, and charge per pixel $q_0$\\
    
    \trigNNc{k}{q_0} & Next neighbour in a closed packet trigger with \\
    &  multiplicity $k=3:5$, and charge per pixel $q_0$\\
  \end{tabular}
\end{center}
%%
In the simulation of the trigger logic, even for \trigNN{k}{q_0} and
\trigNNc{k}{q_0}, a pattern approach has been taken. For a given value
of $k$, and both for \trigNN{k}{q_0} and \trigNNc{k}{q_0} conditions,
we define the set of possible geometric arrangements of triggering
pixels (following the selected trigger condition) in a fictitious
group of seven pixels (a central pixel plus its seven neighbours),
where the charges of the pixels are supposed to be either $0$ or $\geq
q_0$. We give then a name to each of these sets of arrangements.
Finally, in the simulation process, only those patterns in the sets
selected by the user will be searched for in the camera. In
Fig.\ref{fig:trigpatt} we can see a graphical illustration of these
sets of patterns.

\trigpattfig

\MORE%%%%%%%%%%%%%%%%%%%%%%%%%%%%%%%%%%%%%%%%%%%%%%%%%%%%%%%%%%%%

%%------------------------------------------------------------
\section{Results}

After analyzing the raw output data, with the programs
\texttt{reflector} and \texttt{camera}, described above, we obtained
the following results.

%%------------------------------------------------------------
\subsection{Trigger Efficiencies}

The trigger efficiencies for gammas and hadrons are estimated from the
ratio of triggered showers, with a certain trigger criterion, to the
total of incoming showers, for a given set of primary particle, range
in Zenith Angle and core distance. 

For gamma-rays we have the following quantities:
%
\begin{center}
  \begin{tabular}{cl}
    $n_\gamma(r;E,\Theta)$ & number of \emph{incoming} 
    gamma showers arrived with primary energy $E$ \\
    & at an impact parameter $r$ from a zenith angle $\Theta$.\\
    $n_\gamma^T(r;E,\Theta)$ & number of \emph{triggered} 
    gamma showers arrived with primary energy $E$ \\
    & at an impact parameter $r$ from a zenith angle $\Theta$.\\
  \end{tabular}
\end{center}
%
therefore, the \emph{trigger efficiency} (for a given trigger set-up)
for gamma-rays of energy $E$, a zenith angle $\Theta$ and an impact
parameter $r$ is:
%
\effgammaeq

For the background of cosmic rays the situation is more complex. Since
this background comes isotropically, we have a new degree of freedom,
the \emph{angle off-axis}, $\delta$, of the incoming shower with
respect to the optical axis of the telescope. The quantities involved
are now:
%
\begin{center}
  \begin{tabular}{cl}
    $n_{\mathrm{cr}}(r,\delta;E,\Theta)$ & number of \emph{incoming} 
    cosmic ray showers arrived with primary energy $E$ \\
    & at an impact parameter $r$ from a zenith angle $\Theta$
    and an off-axis angle $\delta$\\
    $n_{\mathrm{cr}}^T(r,\delta;E,\Theta)$ & number of \emph{triggered} 
    cosmic ray showers arrived with primary energy $E$ \\
    & at an impact parameter $r$ from a zenith angle $\Theta$
    and an off-axis angle $\delta$\\
  \end{tabular}
\end{center}
%
Thus, \emph{trigger efficiency} (for a given trigger set-up) for
cosmic rays is given by:
%
\effcreq

Using different trigger patterns and different values of the Zenith
Angle $\Theta$, we studied the dependence of the \emph{trigger
  efficiency} with the angle $\delta$ and the impact parameters $r$.
The results are shown in Figs. \ref{fig:effg1}--\ref{fig:effh2}

%%------------------------------------------------------------
\subsection{Effective Collection Areas}

The definition of \emph{effective collection area} is the following:
%
\begin{description}
\item[{\bfseries Effective Collection Area}:] Hypothetic circular area
  in the plane perpendicular to the optical axis of the telescope, at
  his level, where an incoming shower falling inside will be detected.
\end{description}
%
The intuitive idea behind this definition is illustrated in Fig.
\ref{fig:collareaIdea}. In reality, the probability of detection, for
a given energy of the primary (and hence for a given height of
development of the shower), is a more or less smooth function of the
distance to the optical axis of the telescope, and not only a matter
of a yes/no decision. However, this concept is very useful to compare
different detectors.

\collareaIdeafig

For gamma-rays, the calculation of the effective collection area is
very simple. Once we have the trigger efficiency, we calculate, for a
fixed energy $E$ of the primary, and a fixed Zenith Angle $\Theta$:
%
\Sgammaeq
%
The first integral comes from the symmetry around the telescope axis,
and will give just a factor $2\pi$.

For our background of cosmic rays, the situation is now much more
complicated. We are now forced to include our additional degree of
freedom in the calculations. We have
%
\Screq
%
The third integral comes now from symmetry around the telescope axis
of the arrival directions of the incoming showers (giving again a
factor $2\pi$). The fourth integral sums up all the contributions for
diferent off-axis angles $\delta$, up to an angle where the trigger
efficiency is zero (here expressed as $\delta_{\mathrm{max}}$).

Note that this $\hat{S}_{\mathrm{cr}}(E,\Theta)$ has now dimensions
[area$\times$solid angle]. This means that for any comparison with the
effective collection area for gammas we must normalize in solid angle.
The first thing that comes to our mind is to divide by the solid angle
calculated up to the maximum value of $\delta$, that is by
$\Omega_{\mathrm{max}}=2\pi(1-\cos\delta_{\mathrm{max}})$. This is
wrong, since by using an arbitrarily large $\delta^\prime_{\mathrm{max}} \gg
\delta_{\mathrm{max}}$, and hence $\Omega^\prime_{\mathrm{max}} \gg
\Omega_{\mathrm{max}}$, we will barely increase the value of
$\hat{S}_{\mathrm{cr}}(E,\Theta)$, but make the quantity
$\hat{S}_{\mathrm{cr}}(E,\Theta)/\Omega^\prime_{\mathrm{max}}$ arbitrarily small.

%One solution is to estimate the derivative of $\hat{S}_{\mathrm{cr}}$ with
%respect to $\delta_{\mathrm{max}}$, at the point $\delta_0 = 0\deg$,
%%
%\begin{equation}
%  \label{eq:Shadron0}
%  S_{h,0}(E,\Theta) = 
%  \left.
%    \frac{\partial\hat{S}_{\mathrm{cr}}(E,\Theta)}{\partial\delta_{\mathrm{max}}}
%  \right|_{\delta_{\mathrm{max}}=\delta_0}  
%\end{equation}

%One solution is to estimate the value of the trigger efficiency for
%hadrons assuming they all come in the limit $\delta\rightarrow 0\deg$,
%that is
%%
%\begin{equation}
%  \label{eq:limiteff}
%  \varepsilon_{h,0}(r;E,\Theta) = 
%  \lim_{\delta\rightarrow 0\deg} \varepsilon_{\mathrm{cr}}(r,\delta;E,\Theta)
%\end{equation}
%%
%and then use this value to calculate a hypothetic maximum value for
%$\delta$, $\delta_{\mathrm{calc}}$, the working scenario being that we
%get all the showers inside this solid angle, and we have in this
%region a constant trigger efficiency equal to
%$\varepsilon_{h,0}(r;E,\Theta)$ (and therefore, we would end up with
%the same value of $\hat{S}_{\mathrm{cr}}(E,\Theta)$).

We will normalize to the solid angle given by the trigger area in the
camera\footnote{ Note that this does not need to be the same as the
  field of view. Indeed, usually the trigger area is smaller.}. This
normalization takes into account, therefore, the different trigger
efficiency resulting from defining different trigger areas in the
camera. In our standard scenario, having a trigger area given by
$\varphi_{\mathrm{trig}} = 0.8\deg$ (see section
\ref{sec:triggerlogic} above, in particular Eqs.  \ref{eq:trigrad} and
\ref{eq:phitrigger}, and Fig. \ref{fig:trigarea}) this results:
%
\normsolidangleeq
%
and 
%
\Scrnormeq

In this way we calculated the effective collection areas for gammas
and cosmic rays (properly normalized, as we indicated). The results
are shown in Figs. \ref{fig:areagamma}, \ref{fig:areahadr} and
\ref{fig:areacompare}.

\MORE%%%%%%%%%%%%%%%%%%%%%%%%%%%%%%%%%%%%%%%%%%%%%%%%%%%%%%%%%%%%

%%------------------------------------------------------------
\subsection{Detection Rates}

One of our goals is the estimation of the detection rates for gamma
and cosmic rays, mainly at threshold. Let's see how to calculate these
quantities.

Let's assume we are observing a point like source of gamma rays, with
a flux similar to that of our standard candle, the Crab Nebula.
Indeed, we will use as the spectrum for our assumed source of gamma
rays
%
\gFluxeq
%
normalized at $10\u{GeV}$, and extrapolated to higher energies.
This means, a differential flux of
%
\gdFluxeq

For cosmic rays, we used the measured flux:
%
\crFluxeq

We will call \emph{spectral index}, $\alpha$, to the (positive)
exponent of the differential flux. Therefore, we have in our
simulations
%
\specindexeq

%%------------------------------------------------------------
\subsubsection{Differential rates of detection}

With these differential fluxes, we can calculate the differential
detection rates as follows: for gamma rays
%
\gdRateeq
%
where we do not write the dependency in $\Theta$; for cosmic rays
%
\crdRateeq


%%------------------------------------------------------------
\subsubsection{Integral rates of detection}

From the Eq. \ref{eq:gdRate}, the expression for the integral
detection rate for gammas is (forgetting once more the dependency with
$\Theta$; i.e., for a given value of the Zenith Angle)
%
\gRateeq
%
and similarly for cosmic rays
%
\crRateeq


%%------------------------------------------------------------
\subsubsection{Hadronic, electronic and muonic backgrounds}


\randpoifig



\endinput
%
%% Local Variables:
%% mode:latex
%% TeX-master: t
%% End:

%%TODO
%
% Efficiencias de trigger para gammas y hadrones 
%
%
%%TODO


%%EOF


\chapter{Simulaci'on del detector: \MAGIC}
\label{chapter:simmagic}


%------------------------------------------------------------
\section{Reflexi'on y recolecci'on de la luz en el plano focal}
\label{sec:reflcoll}



\randpoifig

\section{Detecci'on en la c'amara}

\subsection{Simulaci'on de la cadena de proceso de la se�al}

\subsection{L'ogica de disparo}

\section{Resultados}

\subsection{Eficiencias de disparo}

\subsection{'Areas de colecci'on efectivas}

\subsection{Ritmos de detecci'on}

\subsubsection{Ritmos diferenciales de detecci'on}

\subsubsection{Ritmos integrales de detecci'on}

\subsubsection{Fondos hadr'onico, electr'onico y mu'onico}

\endinput
%
%% Local Variables:
%% mode:latex
%% TeX-master: t
%% End:

%%EOF
