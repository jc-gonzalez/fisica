%%%%%%%%%%%%%%%%%%%%%%%%%%%%%%%%%%%%%%%%%%%%%%%%%%%%%%%%%%%%%%%%%%%%%%%%%%%
%%
%%  ch-atmshowers.tex
%%
%%  Created: Fri Oct 10 14:24:37 1997
%%  Author.: Jose Carlos Gonzalez
%%  Notes..:
%%          
%%-------------------------------------------------------------------------
%% Filename: $RCSfile$
%% Revision: $Revision$
%% Date:     $Date$
%%%%%%%%%%%%%%%%%%%%%%%%%%%%%%%%%%%%%%%%%%%%%%%%%%%%%%%%%%%%%%%%%%%%%%%%%%%


\chapter{Cascadas atmosf'ericas de rayos gamma y rayos c'osmicos}
\label{chapter:atmshowers}

Las part'iculas de conforman la RADC, a su entrada en la atm'osfera
terrestre interact'uan con las mol'eculas y 'atomos ionizados del
aire, produciendo part'iculas secundarias.  Estas part'iculas
secundarias interact'uan a su vez entre ellas y con otras mol'eculas e
iones de su entorno.  Este proceso, repetido a lo largo de varias
generaciones, da lugar a lo que se denomina una \emph{\I{cascada
atmosf'erica}} (see Fig. \ref{fig:atmshower}).

Obviamente, part'iculas c'osmicas diferentes generar'an diferentes
tipos de cascadas atmosf'ericas. En particular, si la part'icula que
entra en la atm'osfera (\emph{\I{part'icula primaria}}, o
abreviadamente \emph{\Isee{primario}{part'icula primaria}}) es un rayo
gamma ($\gamma$), la composici'on de la cascada ser'a
predominantemente electromagn'etica (fotones, electrones, positrones,
muones, \ldots). Llamaremos a este tipo de cascada \emph{\I{cascada
electromagn'etica}}, o \emph{\I{CEM}}.  Por el contrario, si el
primario es un nucleo at'omico (hadr'on), la componente predominante
de la cascada ser'a hadr'onica, y la denominaremos \emph{\I{cascada
hadr'onica}}, o \emph{\I{CH}}. Aparte del hecho de poseer diferente
primario, ambos tipos de cascadas no se muestran claramente
diferenciadas: por ejemplo, una part'icula $\pi^{0}$ generada en una
interacci'on n'ucleo--n'ucleo se desintegrar'a en un par
$\gamma$-$\gamma$, dando como resultado una sub-cascada de tipo
electromagn'etico.

\atmshowerfig  % artistic view of a Cherenkov experiment (HEGRA?)

\processesfig  % processes involved in generation of gamma rays

Todo lo dicho hasta ahora apnuta, en el caso de observatorios en
tierra, a una medida indirecta de la RADC. Los detectores a nivel del
mar, e incluso a lo alto de monta~nas, no pueden observar las
particulas primarias tal y como llegan a la Tierra. En su lugar, s'olo
pueden detectar las particulas secundarias generadas en el desarrollo
de la cascada en la atm'osfera. Es en este sentido que se dice que la
atm'osfera act'ua como un gran ``calor'imetro''. La consecuencia m'as
importante de esto es que la energ'ia del primario ha de ser inferida
a partir de medidas indirectas, a trav'es de magnitudes relacionadas,
tales como el n'umero total de part'iculas de una determinada especie
(electrones, muones, fotones, \ldots), o su particular distribuci'on
en el suelo. Esto, por supuesto, es un problema, ya que nuestra medida
siempre ser'a menos precisa que mediante la medici'on directa, y
cuanto menos precisa sea este resultado en peor situaci'on estaremos a
la hora de discriminar entre diferentes modelos o escenarios para la
produci'on y el transporte de las part'iculas hasta la Tierra. Por
tanto, es claro que el principal objetivo de nuestros estudios debe
ser la b'usqueda de criterios para la estimaci'on de la energ'ia de
las part'iculas primarias de una manera precisa y fiable. En el camino
hacia el cumplimiento de nuestro objetivo, la simulaci'on de cascadas
atmosf'ericas es de crucial importancia.

%%%%%%%%%%%%%%%%%%%%%%%%%%%%%%%%%%%%%%%%%%%%%%%%%%%%%%%%%%%%
\section{Desarrollo de cascadas atmosf'ericas} 

Antes de entrar en los detalles de la simulaci'on, recordaremos
brevemente la f'isica envuelta en el desarrollo de una cascada
atmosf'erica.  shower.

%%==========================================================
\subsection{Cascadas atmosf'ericas electromagn'eticas}

\toymodelfig
%
Los procesos f'isicos implicados en el desarrollo de cascadas
atmosf'ericas se muestran brevemente en la
Fig. \ref{fig:processes}. Un rayo gamma o un electr'on o positr'on,
puede producir una CEM por \emph{radiaci'on de frenado}
(\emph{bremsstrahlung}) y \emph{producci'on de pares}. Mediante estos
mecanismos, una CEM puede evolucionar desde un primario hasta un
inmenso n'umero de part'iculas. Entre ellas, cada part'icula cargada
puede tambi'en generar fotones \Cerenkov.

Normalmente, uno comienza el estudio de la f'isica de las CEMs
mediante el llamado \emph{modelo de juguete}(\emph{toy-model}). Este
modelo se utiliza para explicar el desarrollo de una CEM, pero su
estructura b'asica puede aplicarse tamb'ien a una CH. El modelo asume
que un rayo $\gamma$ de energ'ia $E_0$, a su entrada en la atmosf'era,
viaja una distancia $\lambda$ antes de crear un par e$^+$e$^-$. Cada
una de estas part'iculas toma la mitad de la energ'ia de la part'icula
original, es decir $E_0/2$. Tras esta primera interacci'on tenemos
pues 2 part'iculas, cada una con la mitad de la energ'ia de la
part'icula original. Estas dos part'iculas viajan de nuevo una
distancia $\lambda$ hasta que sufren \emph{bremsstrahlung} y producen
dos fotones (uno por part'icula) que se quedan con la mitad de la
energ'ia de la part'icula que lo ha producido. Despu'es de este
segundo proceso tenemos 4 part'iculas cada una con una energ'ia
$E_0/4$. Continuando con el desarrollo de la cascada, despu'es $n$
generaciones la cascada ha viajado a lo largo de nua distancia
$X=n\lambda$ (donde $X$ es el espesor atmosf'erico a lo largo a lo
largo del eje de la cascada), y tendremos
%
\toyAeq
%
part'iculas con una energ'ia media por part'icula de
%
\toyBeq

El proceso puede verse de manera esquematizada en la figura  Fig.
\ref{fig:toymodel}.  Ambos mecanismos, \emph{bremsstrahlung} y
\emph{produccion de pares} (o cualquier proceso de desdoblamiento que
tengamos en cuenta en el caso de CHs) contin'ua hasta que
%
\toyCeq
%
siendo $E_{\mathrm{c}}$ la energ'ia cr'itica a la cual estos procesos
son dominados por la \emph{ionizaci'on} y el \emph{scattering Compton}
($E_{\mathrm{c}} \sim 102\u{MeV}$ en la atm'osfera). El n'umero de
part'iculas en el m'aximo de la cascada en este modelo (alcanzado
cuando la energ'ia media de las part'iculas en la cascada es $E\simeq
E_{\mathrm{c}}$) es
%
\toyDeq

\hadronicfig

Las propiedades b'asicas de estas ecuaciones son v'alidas tanto para
CEMs como para CHs. En particular
%
\NXsimpleeq

%%==========================================================
\subsection{Cascadas hadr'onicas}

El comportamiento b'asico de una CH puede ser esbozado utilizando el
mismo modelo usado para CEMs. Un esquema del desarrollo de una CH
puede verse en la Fig. \ref{fig:hadronic}.

Sin embargo, para una primario nucle'onico necesitamos saber la
diferencia entre cascadas generadas por n'ucleos pesados y las
generadas por protones o n'ucleos ligeros. Para simular la presencia
de n'ucleos como part'iculas primarias el modelo m'as simple es el
\emph{modelo de superposici'on} (\emph{superposition model}). En este
modelo se asume que un n'ucleo de masa at'omica $A$ y energ'ia total
$E_0$ es equivalente a $A$ nucleones independentes, cada uno con una
energ'ia $E_0/A$. En este modelo a'un se cumple que
$N(X_{\mathrm{max}}) = E_0 / E_{\mathrm{c}}$, pero la ecuaci'on
\ref{eq:xmax} se transforma en
%
\NXsimpleHadeq
%
En la realidad lo que sucede es que el n'ucleo interact'ua con los
'atomos y mol'eculas de la atm'osfera, y s'olo unos pocos nucleones
son liberados en esta interacci'on y producen piones y otras
part'iculas; el resto del n'ucleo se fragmenta en trozos m'as
peque~nos, quedando por lo general uno de ellos mucho mayor que los
dem'as.

\lognefig

%%==========================================================
\subsection{Part'iculas en la cascada atmosf'erica}

En realidad es preciso realizar un estudio mucho m'as detallado del
desarrollo de una cascada. Utilizando m'etodos de Monte Carlo se
obtiene que el perfil longitudinal medio de una CEM puede describirse
mediante
%
\Neeq
%
donde $N_{\mathrm{e}}$ es el n'umero de electrones en la cascada
(``tama~no de la cascada''), $\tau$ es la profundidad atmosf'erica en
longitudes de radiaci'on, $x=\ln(E_0/E_{\mathrm{c}})$ y $s$ es el
\emph{par'ametro de edad} de la cascada:
%
\ageeq
%
El \emph{par'ametro de edad} $s$ cambia su valor de 0 en el comienzo
de la cascada a 1 en el m'aximo de su desarrollo, y luego hasta 2 en
el punto en el que el n'umero de part'iculas cae por debajo de la
unidad. El la Fig. \ref{fig:logne} se han dibujado varias
distribuciones del desarrollo longitudinal de una CEM.

\samplelonglatdistfig

Para la distribuci'on lateral media de una CEM podemos utilizar la
llamada \emph{f'ormula NKG}
%
\NKGeq
%
donde $\rho(r)$ es la densidad de electrones en funci'on de la
distancia al eje de la cascada $r$, $r_{\mathrm{M}}$ es el \emph{radio
de Moli\`ere} de scattering multiple ($\sim 79\u{m}$ a nivel del mar,
$\sim 106\u{m}$ a 2200\u{m} sobre el nivel del mar), y
%
\fsreq
%
La forma general de la f'ormula NKG ha demostrado ser de gran utilidad
a la hora de ajustar las distribuciones laterales obtenidas en
simulaciones por ordenador. En la Fig. \ref{fig:samplelonglatdist}
podemos ver un ejemplo de distribuciones longitudinales y laterales
medias para CEMs generadas por simulaciones de Monte Carlo, para
energ'ias del primario de 100\u{GeV} y 1\u{TeV}.

%%%%%%%%%%%%%%%%%%%%%%%%%%%%%%%%%%%%%%%%%%%%%%%%%%%%%%%%%%%%
\section{Producci'on de luz \Cerenkov}

La luz \Cerenkov es emitida cuando una part'icula cargada viaja a
trav'es de un medio diel'ectrico con una velocidad mayor que la
velocidad de fase de la luz en ese medio. La condici'on es
%
\Cherenkovcondeq
%
donde $\beta=v/c$, la velocidad de la part'icula respecto de la de la
luz en el vac'io, y $n$ es el 'indice de refracci'on del medio
diel'ectrico (proporcional a la densidad del medio). Esta condici'on
puede ser expresada tambi'en en t'erminos energ'eticos: una part'icula
cargada con una energ'ia superior a una minima energ'ia umbral
$E_{\mathrm{min}}$ producir'a a su paso una polarizaci'on del
medio. El medio entonces emitir'a pulsos que interfieren entre s'i,
resultando en un frente de luz (ver Figs. \ref{fig:cherenkov1} y
\ref{fig:cherenkov2}). La energ'ia minima umbral $E_{\mathrm{min}}$
var'ia con la altura: la raz'on es la variaci'on con la altura del
'indice de refracci'on $n$. La luz producida es emitida con un 'angulo
$\theta$ respecto de la trajectoria de la part'icula cargada. La
distribuci'on de 'angulos resulta estar enormemente picada a un
'angulo $\theta_{\mathrm{c}}$, dado por la condici'on de coherencia de
la luz radiada
%
\lightcoherenceeq%
A nivel del mar, $n \simeq 1.00029$ y $\theta_{\mathrm{c}} \simeq
1.3\deg$. Con la Eq. \ref{eq:cherenkovcond} podemos calcular los
valores de $E_{\mathrm{min}}$ para la emisi'on de luz \Cerenkov en el
aire, para varias part'iculas: para electrones es $21\u{MeV}$, para
muones $4.6\u{GeV}$ y para protones $39\u{GeV}$.

Para entender un poco mejor como evoluciona el frente de luz
\Cerenkov, utilicemos un modelo simplificado de la atm'osfera, el
llamado \emph{atm'osfera estandar} (\emph{standard atmosphere}). En
este modelo la densidad var'ia exponencialmente con la altura
%
\exprhoeq
%
donde $H_{\mathrm{S}} \simeq 7.5\u{km}$ es la llamada \emph{escala de
alturas} (\emph{scale-height}) de la atm'osfera. El \emph{espesor
atmosf'erico} (\emph{atmospheric thickness}) $\tau$ es una medida de
la masa total de aire entre dos puntos (normalmente desde el comienzo
de la atm'osfera hasta una cierta altura). Se calcula integrando la
densidad a lo largo de un determinado camino. Por tanto, en nuestro
modelo
%
\pathinteq

\cherenkov1fig
%
El 'indice de refracci'on del medio viene dado por 
%
\refractionidxeq
%
donde $\tau_0=1030\u{g/cm^2}$ es el espesor atmosf'erico a nivel del
mar y $T=204 + 0.091\tau$ es la temperatura a un determinado espesor
(medido desde el comienzo de la atm'osfera). Por tanto tenemos que
aproximadamente
%
\etaeq
%
con $\eta_0 \simeq 0.00029$. En este modelo, el 'angulo de emisi'on de 
luz \Cerenkov resulta
%
\Cherenkovangleeq
%
y la energ'ia umbral $E_{\mathrm{min}}$
%
\Emineq

Ahora estamos interesados en la cantidad total de luz producida. Si
despreciamos, en aproximaci'on de primer orden, los efectos de
dispersi'on de Coulomb, podemos escribir la p'erdida de energ'ia $E$
por unidad de camino recorrido por una part'icula de carga $e$ y
velocidad $\beta$, en radiaci'on \Cerenkov de longitud de onda entre
$\lambda$ y $\lambda+\d \lambda$ como
%
\dEdheq
%
o, en t'erminos de n'umero de fotones,
%
\dNdheq
%
donde $\alpha = e^2/\hbar c \simeq 1/137$ es la \emph{constante de
estructura fina}, y $n$ se asume constante en el intervalo de
integraci'on. De esta expresi'on podemos deducir el comportamiento
espectral general de las distribuciones:
%
\specdistreq
%
El n'umero de fotones emitidos por un electr'on en un rango espectral
$(\lambda_1,\lambda_2)$ a lo largo de un camino de longitud $l$ es
entonces
%
\phemiteq
%
donde hemos usado la condici'on de coherencia de la luz \Cerenkov
\ref{eq:lightcoherence}. La dependencia $\lambda^{-1}$ implica que la
mayor'ia de los fotones tendr'an longitudes de onda en el
ultravioleta.

La relaci'on obtenida en \ref{eq:cherenkovlight} no significa que
podamos obtener un n'umero virtualmente infinito de fotones
\Cherenkov: en realidad la luz \Cerenkov, generada en el rango del UV, 
es fuertemente absorbida a longitudes de onda por debajo de los
300\u{nm}. Los principales mecanismos responsables de esta absorci'on
son:

\begin{description}

\item[\textbf{Dispersi'on Rayleigh}] (Scattering Rayleigh) Las
  mol'eculas de la atm'osfera producen una dispersi'on de los
  fotones. Este efecto est'a fuertemente correlacionado con la
  longitud de onda de la luz, $\lambda$, siendo m'as importante para
  peque~nos valores de $\lambda$. El coeficiente de transmisi'on
  debido a dispersi'on Rayleigh es
%
  \Rayleigheq
%
  donde $\tau_i = \tau_0 \exp(-h_i/H_{\mathrm{S}}) \sec\theta$ es el
  \emph{espesor atmosf'erico inclinado} (\emph{slant thickness}) por
  encima de una altura $h_i$, $\tau_{\mathrm{R}}(\lambda=400\u{nm}) =
  2970\u{g/cm^2}$ es el \emph{recorrido libre medio} para la
  dispersi'on Rayleigh en t'erminos de espesor, $\tau_0 =
  0.00129\u{g/cm^2}$, y $H_{\mathrm{S}} \simeq 7.5\u{km}$ es la
  \emph{escala de alturas} ya mencionada.
  
\item[\textbf{Dispersi'on Mie}] Este efecto resulta en una dispersi'on 
  casi constante para todas las longitudes de onda. Es debido a
  mol'eculas de aerosoles y polvo suspendidos en el aire. Su efecto es
  m'as pronunciado a bajas altitudes. Tambi'en depende del tama~no de
  las part'iculas. El coeficiente de transmisi'on correspondiente es
%
  \Mieeq
%
  donde $l_{\mathrm{M}}(400\u{nm}) \simeq 14\u{km}$ es el
  \emph{recorrido libre medio} para la dispersi'on Mie, y
  $h_{\mathrm{M}} \simeq 1.2\u{km}$ es la escala de alturas de la
  distribuci'on de aerosoles.
  
\item[\textbf{Absorci'on por Ozono}] Este efecto resulta muy
  importante en el rango de longitudes de onda 280--340\u{nm} y para
  cascadas de baja energ'ia, en las cuales la mayor'ia de la luz
  \Cerenkov es emitida all'i donde la densidad de Ozono es mayor.

\end{description}

\endinput
%
%% Local Variables:
%% mode:latex
%% End:
%%EOF
