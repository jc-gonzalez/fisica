%%%%%%%%%%%%%%%%%%%%%%%%%%%%%%%%%%%%%%%%%%%%%%%%%%%%%%%%%%%%%%%%%%%%%%%%%%%
%%
%%  title.tex
%%
%%  Created: Fri Oct 10 14:43:04 1997
%%  Author.: Jose Carlos Gonzales
%%  Notes..:
%%          
%%-------------------------------------------------------------------------
%% Filename: $RCSfile$
%% Revision: $Revision$
%% Date:     $Date$
%%%%%%%%%%%%%%%%%%%%%%%%%%%%%%%%%%%%%%%%%%%%%%%%%%%%%%%%%%%%%%%%%%%%%%%%%%%
%

%%\def\mititulo{%
%%Simulation of \\%
%%Very High Energy \\%
%%Atmospheric Showers\\%
%%for The MAGIC Telescope\\}

%\def\mititulo{%
%Simulation of \\%
%Atmospheric Showers\\%
%and Performance Studies\\%
%for Cherenkov Telescopes\\}

\def\mititulo{%
Simulaci'on \\%
de Cascadas Atmosf'ericas \\%
y Telescopios Cherenkov \\%
mediante T'ecnicas de Monte Carlo\\}

\thispagestyle{empty}  

\mbox{}

\vskip 50pt

\begin{center}

{\Huge \sffamily \bfseries \sc \mititulo}

\vskip 80pt

{
\large \itshape

Memoria presentada por

\vskip 25 pt

{\Large \bfseries \rm Jos{\'{e}} Carlos Gonz{\'{a}}lez Garc{\'{\i}}a-Consuegra}

\vskip 25 pt

para optar al grado de Doctor en Ciencias F{\'{\i}}sicas.

\vskip 40pt

Dirigida por la profesora

\vskip 25pt

{\Large \bfseries \rm Dra. Mar{\'{\i}}a Victoria Fonseca Gonz{\'{a}}lez}

}

\vskip 45pt

{\Large \rm Septiembre, 2\,000 \\}

\vskip 45pt

{\it \small
Depto. de F{\'{\i}}sica At{\'{o}}mica, Molecular y Nuclear\\
Facultad de Ciencias F{\'{\i}}sicas\\
Universidad Complutense de Madrid\\
} 

\end{center}

\echapter

\thispagestyle{empty}  

\vspace*{2cm}
\begin{flushright}
  \rule{.7\linewidth}{1pt}\\
  \Large \mititulo
  \rule{.7\linewidth}{1pt}
\end{flushright}
\vspace*{6cm}
\begin{flushright}
  \large 
Jos{\'{e}} Carlos Gonz{\'{a}}lez Garc{\'{\i}}a-Consuegra\\[15 mm]
Depto. F{\'{\i}}sica At{\'{o}}mica, Molecular y Nuclear\\
Facultad de Ciencias F{\'{\i}}sicas\\
Universidad Complutense de Madrid\\[15 mm]
Septiembre, 2\,000
\end{flushright}

\echapter

\thispagestyle{empty}  

\vspace*{2cm} 
%
\hfill 
\parbox[b]{0.65\linewidth}{ 
%
\itshape
%
\raggedright
%
There is a theory which states that if ever anyone discovers exactly
what the Universe is for and why it is here, it will instantly
disappear and be replaced by something even more bizarre and
inexplicable. \\
%
\vspace{5pt}
\centerline{---}
\vspace{5pt}
%
There is another theory which states that this has already happened. \\
%
\vspace{15pt}
%
\upshape
%
\raggedleft
%
{\footnotesize (Douglas Adams, ``The Restaurant at the End of the
  Universe'')}
%
} 

\echapter

\endinput
%
%% Local Variables:
%% mode:latex
%% End:

%%EOF
