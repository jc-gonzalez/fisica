%%%%%%%%%%%%%%%%%%%%%%%%%%%%%%%%%%%%%%%%%%%%%%%%%%%%%%%%%%%%%%%%%%%%%%%%%%%
%%
%%  ch-magic.tex
%%
%%  Created: Fri Oct 10 14:24:37 1997
%%  Author.: Jose Carlos Gonzalez
%%  Notes..:
%%          
%%-------------------------------------------------------------------------
%% Filename: $RCSfile$
%% Revision: $Revision$
%% Date:     $Date$
%%%%%%%%%%%%%%%%%%%%%%%%%%%%%%%%%%%%%%%%%%%%%%%%%%%%%%%%%%%%%%%%%%%%%%%%%%%

\chapter{El Telescopio \MAGIC}
\label{chapter:magic}

El Telescopio \I{MAGIC} (\emph{Major Advanced Gamma-rays Imaging
  Cherenkov telescope}) es un projecto cuyo objetivo es la
construcci'on de un telescopio \Cerenkov de 17\u{m} de di'ametro, con
la tecnolog'ia m'as avanzada disponible en este campo. Muchos de los
nuevos elementos utilizados han sido especialmente desarrollados para
este proyecto, como por ejemplo espejos ultraligeros y de alta
reflectividad de aluminio anodizado, control activo de los espejos
(\I{active mirror control}), \I{Flash ADC}s, conexiones de \I{fibra
  'optica}, mecanismo de posicionamiento r'apido para la observaci'on
de \I{GRB}s, sistema de disparo (\I{trigger}) multinivel, monitoraje
de la atm'osfera, etc.

Hoy en d'ia los experimentos para la detecci'on de rayos $\gamma$ a
bordo de sat'elites pueden operar hasta energ'ias de 10\u{GeV}. Estos
dispositivos son sin embargo ineficientes a la hora de detectar
energ'ia superiores, debido a su peque~na area de colecci'on. Por otro
lado, los m'as avanzados detectores \Cerenkov operativos actualmente
en tierra tienen un umbral de detecci'on de unos 200--300\u{GeV}. Por
debajo de esta energ'ia su sensibilidad cae drasticamente. El
principal objetivo de \MAGIC es la cobertura del rango energ'etico
entre 10\u{GeV} y 200\u{GeV}.
%
\energygapfig

El detector \I{EGRET} (\emph{Energetic Gamma Ray Experiment
  Telescope}, instalado a bordo del sat'elite \I{CGRO} (\emph{Compton
  Gamma-Ray Observatory}, con un umbral superi'or de detecci'on de
10\u{GeV}, registr'o datos de un gran n'umero de fuentes emisoras de
rayos $\gamma$. Sin embargo, la mayor'ia de las fuentes
extragal'acticas detectadas a alto %
%
$z$\footnote{La magnitud $z$ es el llamado \emph{desplazamiento al
    rojo} (\emph{redshift}), y especifica la variaci'on en longitud de
  onda de una onda proveniente de una fuente que se aleja de nosotros.
  Su c'alculo se puede realizar utilizando la expresi'on
  $\Delta\lambda/\lambda=v/c=z$, donde $v$ es la velocidad de la
  fuente y $c$ es la velocidad de la
luz en vac'io.}%
%
no pueden ser observadas a energ'ias por encima de unos cientos de
\u{GeV} por otros detectores. Adem'as, a energ'ias de varios \u{TeV}
hay s'olo tres remanentes de supernova (\emph{\Isee{supernov{\ae}\ 
    remnant}{SNR}}, \I{SNR}) observadas: la Nebulosa del Cangrejo
(Crab Nebula), Vela y \I{SN\,1006}, incluso cuando se supone que los
SNR son una de las principales fuentes de rayos c'osmicos. En ambos
casos (fuentes extra-gal'acticas y supernovas) la detecci'on parece
posible en el rango 10--200\u{GeV}. Por tanto, es claro que esta
ventana de energ'ias es de extrema importancia para la astronom'ia de
rayos $\gamma$.

\CGROenergiesfig

\section{Objetivos f'isicos}
%
Podemos decir que el principal objetivo del proyecto \MAGIC es la
\emph{cobertura del rango de energ'ia a'un inexplorado de 10\u{GeV} a
  100\u{GeV}}. Este objetivo global viene expresado en una gran
variedad de objetivos espec'ificos, que tratar'e de enumerar:

\subsubsection*{N'ucleos de galaxias activas (Active Galactic 
Nuclei--AGN)}
%
Alrededor del 1\% de las galaxias son activas. \emph{Activa} significa,
en este contexto, que su n'ucleo tiene asociado una emisi'on de
car'acter no t'ermico altamente luminosa y variable.  En esta clase de
objetos nos encontramos con las radio-galaxias, las galaxias Seyfert,
los \emph{\I{cuasares}} (radio-objetos \emph{cuasi-estelares}, con
emisi'on radio tanto intensa como d'ebil) y los objetos
\emph{\I{BL-Lacert{\ae}}} (\I{BL-Lac}). En la figura
\ref{fig:AGNclassification} se muestra una clasificaci'on de la
familia de los AGN (tomado de \cite{Petry:tesis}).

%\AGNclassificationfig

Desde la detecci'on por EGRET de varios AGN, la atenci'on se ha
centrado en la subfamilia \emph{blazar}, a la que pertenecen.  Esta
subclase esta integrada por los \emph{\I{objetos BL Lac}} y los
\I{FSRQ}s (\emph{Flat Spectrum Radio Quasars}, radio-cuasares de
espectro plano). Ambos tipos de objetos vienen caracterizados por un
espectro plano en radio, fuerte variabilidad y polarizaci'on de su
emisi'on en el 'optico. Los blazares son los principales candidatos
para la emisi'on gamma de alta y muy alta energ'ia.

Las principales cuestiones relacionadas con la observaci'on de AGN
son:
%
\begin{itemize}
\item ?`Como se forman y evolucionan los AGN?
\item ?`En que consisten las ejecciones de material observadas
  (\emph{jets}, emisiones de plasma relativista) y como es este plasma
  acelerado?
\item ?`Hay un corte brusco (\emph{cut-off}) en el espectro de
  energ'ias de los fotones? ?`C'omo y donde se producen estos
  \emph{cut-offs}?
\item ?`Cu'al es la raz'on para tan fuerte variabilidad?
\end{itemize}

Si seleccionamos los objetos BL Lac con emisi'on en rayos X, a bajas
energ'ias (hasta unos 1--100\u{keV}) la emisi'on es con toda
probabilidad debida a la emisi'on sincrotr'on de un haz de electrones
altamente relativistas. Por encima de 300\u{GeV}, s'olo tres AGN han
sido detectados (unos de ellos s'olo por el grupo del telescopio
Whipple, en una corta emisi'on espor'adica intensa). Pero a'un m'as
importante es el hecho de que s'olo uno de ellos (Mkn 421) ha sido
detectado por EGRET. Puesto que EGRET ha detectado varias decenas de
AGN a energ'ias por debajo de 30\u{GeV}, la mayor'ia de estos AGN
deben tener un corte brusco (\emph{cut-off}) en su espectro a
energ'ias entre 30 y 300\u{GeV}. Existen varios escenarios con
explicaciones diferentes.  En cualquier caso, si asumimos que el fondo
extra-gal'actico de radiaci'on $\gamma$ difusa se debe a la suma de
las emisiones de blazares no resueltos, el espectro de estos AGN
deber'ia extenderse sin cambio dr'astico en la pendiente al menos
hasta los 50--100\u{GeV}.

Podr'ia tambi'en existir incluso una subclase de AGN con su segundo
pico de la distribuci'on espectral de energ'ia por encima de los
10\u{GeV} (el primer pico del espectro aparece entre el UV y los rayos
X). La detecci'on de estos AGN por sat'elite ser'ia dif'icil. Todo
esto hace el estudio de los AGN en este rango de energ'ias
especialmente interesante para la construcci'on de modelos de los
blazares.

\subsubsection*{Remanentes de supernova (SNRs)}
% 
Como ya se mencion'o, solo tres SNRs han sido asociados a emision de
rayos $\gamma$: la Nebulosa del Cangrejo, Vela y SN1006. Se piensa que
el mecanismo de producci'on de rayos $\gamma$ es el efecto Compton
inverso (\emph{Inverse Compton Scattering}, ICS) producido por
electrones de muy alta energ'ia sobre fotones de su ambiente y fotones
sincrotr'on, y la desintegraci'on de part'iculas $\pi^0$ en $\gamma$s
despu'es de la interacci'on de hadrones acelerados con la materia
externa al SNR. Las principales cuestiones de inter'es para el estudio
de los SNRs con \MAGIC son:
%
\begin{itemize}
\item ?`Qu'e tipo de part'iculas son las responsables de la
  producci'on de $\gamma$s de alta y muy alta energ'ia en los SNRs?
\item ?`Es el mecanismo de producci'on de $\gamma$s dependiente del
  tipo de SNR?
\item ?`Son los SNRs realmente los responsables del grueso de rayos
  c'osmicos observados?
\end{itemize}

\subsubsection*{P'ulsares} 
%
En los SNRs del tipo \emph{pleri'on} la fuente energ'ia son los
radio-p'ulsares. Adem'as del pulsar del Cangrejo, se han detectado
otros dos p'ulsares a muy altas energ'ias: el pulsar de Vela
(pleri'on) y PSR\,1706-44. Todos los p'ulsares detectados por EGRET
parecen tener propiedades bastante dispares. En la actualidad los
f'isicos te'oricos sugieren que no es esperable la emisi'on pulsada de
p'ulsares aislados por encima de 6--20\u{GeV}, en el marco del
\emph{modelo del casquete polar} (\emph{polar-cap model}), o de los
100\u{GeV} en el \emph{modelo del hueco externo} (\emph{outer-cap
model}). 

Respecto de los p'ulsares sin emisi'on en radio, solamente uno ha sido
detectado a altas energ'ias: el pulsar Geminga. Asumiendo que este es
el p'ulsar m'as cercano de su clase se ha podido estimar el n'umero
de objetos de este tipo en unos 1600. \MAGIC buscar'a emisi'on de
objetos m'as d'ebiles que Geminga. En esta clase de objetos se
encuentran tambi'en las m'as d'ebiles de las fuentes no identificadas
del cat'alogo de EGRET.

\subsubsection*{\emph{Gamma Ray Bursts}} 
%
Uno de los misterios mejor guardados hasta ahora por la Naturaleza en
el campo de la astrof'isica de rayos gamma es la naturaleza de los
\emph{estallidos de rayos $\gamma$} (\emph{gamma ray bursts}, GRBs).
Con \MAGIC se espera poder medir el espectro de energ'ias de los GRBs.
\MAGIC ser'a capaz de realizar un posicionamiento ra'pido (cambio de
apuntado entre dos puntos cualesquiera del cielo en menos de 30
segundos). Esto, junto con la informaci'on sobre posici'on y tiempos
obtenida de futuros detectores a bordo de sat'elites, nos permitir'a
obtener datos sobre la mayor parte de la emisi'on proveniente de los
GRBs (el grueso de la emisi'on dura alrededor de 100\u{s}, aunque a
veces puede durar m'as --- para el GRB\,971208, por ejemplo, el
estallido observado consist'ia en un 'unico pico que dur'o alrededor
de 800\u{s}).

\subsubsection*{Fuentes de EGRET no identificadas} 
%
Ya hemos mencionado alguna vez la importancia de observar las fuentes
no identificadas del cat'alogo EGRET en el rango
10--200\u{GeV}. Alrededor del 40\% de estas fuentes poseen un espectro 
duro que parece extenderse mas all'a de los 10\u{GeV}. La observaci'on 
de estos objetos con \MAGIC aumentar'a la precisi'on de su
localizaci'on de 0.1\deg a 0.02\deg, y por tanto la posibilidad de su
final identificaci'on.

\MAGICframefig

\section{Caracter'isticas 'opticas del reflector}
%
El telescopio \MAGIC posee un espejo primario de 17\u{m} de di'ametro.
Su montura es alt-azimutal. El espejo reflector est'a soportado por
una estructura tubular en tres capas, que a su vez est'a fijada al eje
azimutal en dos puntos. Esta estructura tubular est'a formada for
tubos de fibra de carbono y epoxy ultraligeros de alta dureza,
ensamblados mediante piezas (nudos) de aluminio. El soporte del espejo
est'a montado en el interior de un anillo, el anillo $\theta$, que
soporta la c'amara en su 'apex superior, mientras que la secci'on
inferior se encuentra equipada por una gu'ia para el movimiento el
elevaci'on (ver Fig.\ref{fig:MAGICframe}). El telescopio ha sido
dise~nado para tener una 'optica f/D:1, es decir la distancia focal
global del sistema es de 17\u{m}. El anillo $\theta$ tiene forma
el'iptica (\emph{arco g'otico}), con objeto de evitar los 30--32\u{m}
de di'ametro que deber'ia tener un anillo circular que cumpliese con
el requisito f/D:1.

La forma global del disco del telescopio es paraboloidal, y su espejo
es teselado: est'a compuesto de 920 peque~nos espejos cuadrados de
$50\times 50 \u{cm}^2$ cada uno, de superficie esf'erica. Este dise~no
permite una \emph{aberraci'on esf'erica} peque~na junto con una
dispersi'on temporal del pulso \Cerenkov m'inima. Dada la gran
variaci'on de distancia desde el eje del telescopio hasta cada espejo,
'estos se han agrupado en conjuntos de radio de curvatura com'un,
variando este radio de curvatura de 34\u{m} a 36\u{m} aproximadamente.
Los espejos se encuentran dispuestos en filas y columnas, y cada
conjunto de $2 \times 2$ espejos se encuentra montado sobre un
elemento individual de $1 \times 1 \u{m^2}$. Cada uno de estos
elementos individuales de 4 espejos se ajustar'a y enfocar'a de manera
independientemente (\emph{'optica activa}).

La deformaci'on de la estructura que soporta los espejos ser'a
monitorizada utilizando haces de luz l'aser que incidan en sensores de
silicio semi-transparentes, con una resoluci'on global en la medida de
las deformaciones de unos pocos micr'ometros (\u{\mu m}). Para la
\emph{'optica activa} se utilizar'a un l'aser localizado en cada uno
de los paneles de cuatro espejos, y que apuntar'a al centro de la
c'amara. De cada panel se tomar'a una imagen del punto de luz en el
plano focal con una video c'amara, y dos motores de pasos
(\emph{stepping motors}), controlados por un multiplexor por un
circuito controlado por computadora, ajustar'an la posici'on y
orientaci'on del panel en cuesti'on (ver Fig.
\ref{fig:MAGICactivemirror}).

\MAGICmirrorcontrol_wfig

Los espejos individuales poseen diferentes distancias focales, y por
tanto sus superficies han de ser pulidas por una m'aquina controlada
num'ericamente. \MAGIC usar'a espejos en forma de sandwich de
aluminio, similares a los ya usados por el telescopio CT1 de HEGRA
(ver \ccite{MAGIC:Barrio_Kruger}). En la Fig. \ref{fig:mirrorsandwich}
podemos ver los elementos internos de uno de estos espejos. La
reflectividad de la superficie especular, hecha de una aleaci'on de
aluminio, es del orden de $R(\lambda) > 80\%$ para el rango de
longitudes de onda $320--650\u{nm}$. Adem'as, los espejos est'an
protegidos con un recubrimiento duro transparente.

\section{La c'amara de \MAGIC}
%
\subsection{Generalidades}
%
La c'amara es el elemento decisivo en la construcci'on de todo
telescopio \Cherenkov. Como dispositivo detector, sus caracter'isticas
tendr'an una gran influencia en el comportamiento global del
telescopio, en t'erminos de eficiencia, sensibilidad y rango din'amico
(en el sentido de energ'ias umbral y de saturaci'on, por ejemplo). Hoy
en d'ia, las c'amaras de los telescopios \Cherenkov operativos est'an
compuestas por peque~nos elementos detectores (\emph{p'ixeles}),
normalmente \emph{\I{photomultiplicadores}} (PMTs). Por tanto, las
car'acter'isticas de un telescopio \Cherenkov pueden, al menos en
parte, ser atribu'idas a:
%
\begin{enumerate}[a.]
\item La distribuci'on de los p'ixeles en la c'amara (geometr'ia)
\item El tama~no de los p'ixeles individuales
\item El tama~no angular de la c'amara (campo de visi'on --- c.d.v.)
\item La densidad variable de p'ixeles en diferentes partes de la
  c'amara (relacionada con el tama~no de los p'ixeles)
\item La sensibilidad de los p'ixeles
\end{enumerate}

\paragraph{La \emph{distribuci'on de los p'ixeles en la c'amara}} 
afecta s'olo ligeramente el comportamiento del telescopio. Mientras
una distribuci'on en forma de filas y columnas puede resultar m'as
sencilla de construir, e incluso m'as conveniente desde el punto de
vista del an'alisis de datos, la distribuci'on universalmente
utilizada es en forma de \emph{patr'on hexagonal}. Utilizando esa
disposici'on con p'ixeles hexagonales uno consigue minimizar los
espaci'on muertos entre p'ixeles.

\paragraph{El \emph{tama~no de los p'ixeles}} se va reduciendo poco a
poco seg'un avanza el desarrollo tecnol'ogico. Por ejemplo, la primera
c'amara de un telescopio \HEGRA ten'ia un tama~no angular de p'ixel de
0.4\deg, mientras en la siguiente generaci'on de telescopios
\Cherenkov el tama~no de los p'ixeles ser'a de 0.10\deg a 0.15\deg.
Consideraciones te'oricas dan un valor 'optimo para el tama~no de los
p'ixeles de $\leq 0.1\deg$. Para \MAGIC planeamos el uso de dos
tama~nos diferentes de p'ixeles en la c'amara. En la parte central de
la c'amara usaremos p'ixeles de 0.1\deg de di'ametro. Rodeando a la
zona central de la c'amara tendremos anillos adicionales de p'ixeles
de 0.2\deg. La raz'on para esta distribuci'on es la siguiente: los
rayos $\gamma$ de baja energ'ia producen cascadas cuyas im'agenes,
comparadas con las de rayos $\gamma$ de alta energ'ia, se encuentran
m'as cerca del centro de la c'amara (ya que se producen m'as alto en
la atm'osfera) y son m'as compactas (la distancia desde el telescopio
a la cascada es mayor, la cantidad de luz es menor y el 'angulo
\Cherenkov en el punto del m'aximo de desarrollo es menor que para
cascadas con primarios m'as energ'eticos). Todo ello apunta, por
tanto, a la necesidad de usar en el centro de la c'amara p'ixeles
peque~nos. Como los efectos de aberraci'on aumentan conforme nos
alejamos del centro de la c'amara, y como las cascadas de alta
energ'ia dar'an trigger incluso cuando su imagen se encuentra alejada
de ese punto (ya que producen mucha m'as luz), en la zona exterior de
la c'amara no necesitamos una pixelizaci'on muy fina. Esto justifica
el uso de p'ixeles mayores con el fin de minimizar costes.

\mirrorsandwichfig

\paragraph{El \emph{tama~no angular de la c'amara} (campo de visi'on 
  --- c.d.v.)}, por otro lado, va aumentando progresivamente. Los
beneficios que aporta un gran c.d.v son:
%
\begin{enumerate}[i)]
\item El estudio de \emph{fuentes extensas} resulta practicable.
%
\item La reconstrucci'on de cascadas de alta energ'ia es m'as
  eficiente. Para una c'amara de tama~no reducido tenemos lo que se
  denomina \emph{efectos de borde}: una imagen suficientemente grande
  quedar'a truncada en el borde, y por tanto el c'alculo de los
  par'ametros de la imagen y la cantidad de luz producida resultar'an
  afectados, dando como resultado una mala estimaci'on de la energ'ia
  del primario, e incluso el rechazo de sucesos genuinos generados por
  rayos $\gamma$.
%
\item A \emph{alto 'angulo cenital} (AAZ) las cascadas se desarrollan
  a distancias del telescopio mayores que para \emph{bajo 'angulo
    cenital} (BAZ). Por tanto, debido a la dispersi'on y el momento
  transverso de las part'iculas cargadas, el \emph{pool}
  \Cherenkov\footnote{El \emph{pool} \Cherenkov es la regi'on en el
    suelo donde se encuentra la mayor densidad de luz \Cherenkov,
    desde el eje de la cascada hasta la posici'on del \emph{hump}.
    Dependiendo de la energ'ia del primario, en esta regi'on la
    densidad de luz puede aumentar, disminuir o permanecer
    aproximadamente constante con la distancia al eje de la cascada;
    pero se puede asumir, en las energ'ias en las que nos movemos, que
    su distribuci'on es bastante uniforme. Mas all'a del \emph{hump},
    la densidad de luz cae exponencialmente.} es mucho mayor. La
  consecuencia de esto es que las cascadas pueden dar activar la
  l'ogica de \emph{trigger} en el sistema hasta mayores par'ametros de
  impacto. Pero esto significa que la imagen de la cascada aparecer'a
  m'as alejada del centro de la c'amara que en el caso de BAZ. Por
  tanto, para las observaciones a AAZ es necesaria una c'amara grande.
\end{enumerate}

\paragraph{La \emph{sensibilidad de los p'ixeles}} es sin duda uno de
los factores m'as importantes en el desarrollo de la c'amara. El
rendimiento de todo el sistema es directamente proporcional a la
\emph{Eficiencia Cu'antica} de los p'ixeles (\emph{quantum efficiency}
en ingl'es, de ahora en adelante \QE). Hablaremos de los diferentes
tipos de p'ixeles mas adelante.

\MAGICcamerafig

Para \MAGIC utilizaremos una c'amara de unos $\sim$ 4\deg--5\deg, con
un total de alrededor de 900 p'ixeles. 
%(En las simulaciones, sin
%embargo, s'olo hemos utilizado una c'amara de 3.5\deg, con todos los
%p'ixeles hexagonales del mismo tama~no, 0.1\deg de di'ametro). 
En la Fig. \ref{fig:MAGICcamera} podemos observar una de las posibles
distribuciones de los p'ixeles en la c'amara de \MAGIC (dise~no del
autor, adaptado de \ccite{MAGIC:DR}).

\QuantumEfffig

\subsection{Sensores de luz}
%
En el dise~no de \MAGIC se tuvo en cuenta la posibilidad de una
c'amara extraible e intercambiable por otra de diferente dise~no. El
objetivo es aumentar la modularidad del sistema y disminuir la
complejidad de las actualizaciones sobre sistema. Se han dise~nado,
estudiado y probado varios dispositivos de detecci'on; esto ha dado
como resultado tres diferentes concepciones de la c'amara:
%
\begin{enumerate}[a.]
\item La \emph{c'amara cl'asica}, que utiliza fotomultiplicadores est'andar
%
\item La \emph{c'amara estandar}, que utiliza fotodetectores h'ibridos:
fotoc'elulas intensificadas, de Ga\,As\,P, de alta eficiencia (alta
\QE), con lectura por diodos de avalancha
%
\item La \emph{c'amara avanzada}, una soluci'on futura, que emplear'a
fotodiodos de avalancha de Silicio
\end{enumerate}
%
Colocaremos conos huecos (\emph{conos de Winston}) encima de cada
p'ixel con objeto de concentrar la luz al m'aximo en el dispositivo
detector, y restringir la aceptancia de cada p'ixel. Adem'as, se
usar'an \emph{recubrimientos de desplazamiento de longitud de onda}
(WLSs, del ingl'es \emph{wavelength shifters}) para aumentar la
detecci'on de la luz \Cherenkov en el UV, mediante el desplazamiento
de parte de esta luz a zonas del espectro con mayor longitud de onda,
donde la \QE pueda ser mayor (ver Fig. \ref{fig:QE}). Hablaremos un
poco de estos diferentes dispositivos de detecci'on en los siguientes
p'arrafos.

\pixelreadoutfig

\subsubsection{Dispositivos cl'asicos: fotomultiplicadores (PMTs)}
%
En una primera fase del proyecto se utilizar'a una c'amara compuesta
de fotomultiplicadores (a los que denominaremos PMTs, del ingl'es
\emph{photo-multipliers}) est'andar. Esto afectar'a no s'olo al umbral
de energ'ia del sistema, sino tambi'en a la eficiencia en las
observaciones a AAZ. Sin embargo, a'un con este tipo de c'amara la
disminuci'on en energ'ia umbral respecto de los telescopios operativos
actualmente es considerable: nuestras simulaci'on lo colocan en torno
a los 30--40\u{GeV}. Para las simulaciones presentadas en este trabajo
se ha utilizado como PMT el EMI 9083-A (ver Fig. \ref{fig:QE}), que
posee un pico de \QE de cerca del 26\% y una \emph{\QE efectiva
media}%
\footnote{Definimos la \emph{\QE efectiva media}, 
$\langle\QE\rangle_\mathrm{eff}$ como el valor medio de la \QE en el
rango de longitudes de onda 290--600\u{nm} (nuestro rango de trabajo),
convolucionada con el espectro de luz \Cherenkov, es decir:\\
\centerline{$ \langle\QE\rangle_\mathrm{eff} = 
\frac{\int_{\lambda_\mathrm{min}}^{\lambda_\mathrm{max}} \QE(\lambda)
I_\mathrm{Cherenkov} (\lambda) \mathrm{d}\lambda }%
{\int_{\lambda_\mathrm{min}}^{\lambda_\mathrm{max}}
I_\mathrm{Cherenkov} (\lambda) \mathrm{d}\lambda} $} } de alrededor del
17\%. %$

\subsubsection{Fotoc'elulas h'ibridas con intensificaci'on}
%
Para cumplir nuestro objetivo de bajar el umbral de detecci'on en
energ'ia hasta los 10\u{GeV} nos proponemos utilizar en un futuro
cercano las fotoc'elulas h'ibridas con intensificaci'on (IPCs, de
\emph{intensified photo-cells}, nombre dado por la compa~nia Intevac,
uno de los desarrolladores.  Los IPCs alcanzan su m'aximo de m'as de
40\% \QE en algunos puntos de la nuestra region de inter'es
290--600\u{nm}, existiendo a'un posibilidad de mejoras. Este hecho
junto con el uso de WLSs incrementar'a significativamente la
$\langle\QE\rangle_\mathrm{eff}$ para este tipo de c'amara.  Mediante
el uso de fotodiodos de avalancha (en lugar del cl'asico diodo) en el
'anodo, se puede conseguir respuestas de fotoelectrones individuales.
Adem'as, el problema de la \emph{realimentaci'on por iones}
(\emph{ion-feedback}, en ingl'es --- iones liberados en la vecindad
del 'anodo, y que son acelerados de vuelta hacia el c'atodo) puede ser
prevenido con el uso de un deflector de iones.

\subsubsection{Fotodiodos de avalancha}
%
La version avanzada de una futura camar'a de \MAGIC usar'a la
tecnolog'ia m'as novedosa disponible en el momento, los
\emph{fotodiodos de avalancha} (APDs, del ingl'es \emph{avalanche
  photo-diodes}). Estos elementos detectores alcanzan valores de \QE
de $\geq$80\%. Este valor es el doble del mejor valor obtenido nunca
con un IPC. Los APDs permitir'an adem'as la construcci'on de p'ixeles
m'as peque~nos, del orden de (0.05\deg--0.07\deg), ya que su tama~no
es de 3--4\u{mm}. Sin embargo, la tecnolog'ia necesaria para su
utilizaci'on en este campo no est'a a'un lista. El principal problema
de los APDs es su gran nivel de ruido, a'un inaceptable para nuestros
prop'ositos.  No obstante, el desarrollo contin'ua, y est'a previsto
que en pocos a~nos sea posible la construcci'on de la c'amara avanzada
para \MAGIC.

\bigskip

Todos estos dise~nos de la c'amara comparter una cadena de lectura de
datos com'un, que se muestra en la Fig. \ref{fig:pixelreadout}.

%\MORE%%%%%%%%%%%%%%%%%%%%%%%%%%%%%%%%%%%%%%%%%%%%%%%%%%%%%%%%%%%%

\section{Otros componentes del sistema}
\label{sec:othercomp}

\subsubsection{Monitor de deformaci'on de la montura}
%
El soporte del espejo de \MAGIC se compone de una estructura en tres
capas de tubos de fibra de carbono. Esta construcci{\'o}n hace que \MAGIC
posea una estructura muy r'igida. Sin embargo, cuando el telescopio se
mueve, es inevitable que aparezcan deformaciones. Por tanto, se
necesita un sistema para la monitorizaci'on de estas deformaciones,
que nos proveer'a de la informaci'on necesaria para el estudio de
'estas en funci'on de diversas condiciones (orientaci'on de la
montura, viento, \ldots). Se planea \cite{MAGIC:DR} utilizar sensores
semitransparentes de micropistas de silicio en cada una de las 8
esquinas del plato octogonal, y dos l'aseres encima de ambos lados del
eje horizontal (cenital), uno para cada mitad del plato. Una
deformaci'on del plato implicar'a un desplazamiento de los sensores
con respecto de los haz l'aser. Cada sensor registrar'a este
desplazamiento y enviar'a la informaci'on al control central.

\subsubsection{Monitor de seguimiento}
%
Se prevee el uso de dos sistemas para la monitorizaci'on de la
posici'on del telescopio durante el seguimiento de una fuente. Por un
lado, se montar'an dos servo-posicionadores de c'odigo ``Gris'' de 14
bits an ambos ejes cenital y azimutal. La resoluci'on de estos
posicionadores, 0.02\deg, es suficientemente menor que el tama~no de
los p'ixeles.

Adem'as, se usar'a un visor CCD de 10\deg de campo de visi'on y alta
sensibilidad ($\sim 0.01\u{lux}$) para controlar la posici'on de la
c'amara de \MAGIC: registrar'a la posici'on relativa de la c'amara
(con ayuda de unos diodos LED que actuar'an de marcadores) con
respecto de las estrellas en el fondo del cielo. La salida del CCD se
enviar'a a una tarjeta garbadora de im'agenes, instalada en un
ordenador. Esto permitir'a registrar la informaci'on acerca de la
posici'on de la c'amara a la vez que los datos de los sucesos.

Probablemente se usar'a una segunda c'amara CCD para el control de la
calidad del cielo en la zona dentro del campo de visi'on del sistema.

\subsubsection{Control activo de espejos}
%
Cada uno de los soportes donde van montados 4 espejos pueden
orientarse de manera individual. Esto es 'util por ejemplo para
cambiar la distancia focal del sistema, o para el reajuste de las
orientaciones de los espejos (incluso de manera casi simult'anea en el
curso de unas observaciones). Teniendo esto en mente, se ha pensado en
un sistema de control activo de los espejos. 'Este ya fue descrito en
la p'agina
\pageref{label:amcs}. En pocas palabras, cada panel soporta un
peque~no emisor l'aser en su centro, as'i como tres tornillos de
ajuste: uno de ellos es una junta universal de doble precisi'on, y
permite movimientos en el eje XY; los otros dos se encuentran
asociados a sendos motores de pasos. Uno de estos motores se encuentra
permanentemente alineado, sin posibilidad de movimiento en el plano
XY, mientras que el otro puede bascular en una direcci'on. Ambos
motores de pasos son dirigidos mediante via un multiplexador por un
circuito controlado por ordenador.

Para ajustar la orientaci'on de un panel, el puntero l'aser se
enciende, y proyecta un punto de luz en una pantalla blanca en la
posici'on de la c'amara (se podr'ia usar incluso la propia cubierta de
la c'amara en posici'on cerrada). Una videoc'amara CCD registra el
desplazamiento del punto de luz respecto del centro de referencia de
la pantalla, y el sistema corrige consecuentemente la orientaci'on del
panel. Este procedimiento se repite para todos los paneles, uno por
uno.  Los primeros tests \cite{MAGIC:Wacker_Kruger} indican un tiempo
de unos \u{s} para ajustar un s'olo panel.  Por tanto, ser'ian
necesarios unos 20\u{min} para el ajuste de todos los espejos. Sin
embargo, es muy probable que una gran cantidad de espejos no
necesiten ajustes: an'alisis realizados mediante elementos finitos
muestran que una gran 'area de la zona central del reflector no
experimenta deformaci'on apreciable. Este hecho, junto con la
posibilidad de usar listas de orientaciones pre-grabadas, apuntan a
que el procedimiento de apuntado en su conjunto podr'ia realizarse en
muy pocos minutos.

%Un procedimiento alternativo ser'ia el uso de una fuente de luz
%distante (una l'ampara, una estrella, \ldots). Mover'iamos el panel a
%ajustar, primero hacia un lado hasta desplazar la imagen de la fuente
%de luz un 'angulo fijo predeterminado (por ejemplo 1\deg), y luego
%hacia el lado opuesto hasta desplazar la imagen del centro el mismo
%'angulo. Esto ha de repetirse varias veces, cambiando la orientaci'on
%del movimiento basculante descrito. La posici'on central se supone que
%corresponder'a con la orientaci'on 'optima de los paneles, y puede ser
%obtenida por interpolaci'on. Este procedimiento se encuentra libre de
%posibles alineaciones erroneas del puntero l'aser, pero no puede ser
%f'acilmente implementada en las rutinas de correcci'on.

\subsubsection{Control de calidad de los espejos}
%
Las caracter'isticas de los espejos individuales ya fueron estudiados
en la p'agina \pageref{label:mirrorelement}. En la fig.
\ref{fig:MAGICmirrorsandwich} puede apreciarse una vista por capas de
uno de estos espejos.

Se ha exigido una reflectividad para los espejos de $R(\lambda)>85\%$, siendo
$\lambda$ la longitud de onda, en la regi'on espectral 320--650\u{nm}.  Con
objeto de prevenir una grave degradaci'on de los espejos es necesario
el uso de un sistema de monitorizaci'on.  La reflectividad es uno de
los factores que m'as directa incidencia tienen en el factor de
conversi'on fot'on-fotoelectr'on del sistema, y por tanto en la
determinaci'on de la energ'ia de los primarios.  Hasta ahora, con el
sistema de telescopios de \HEGRA, se hab'ia usado un reflect'ometro de
mano.  Con este dispositivo uno pod'ia comprobar uno a uno todos los
espejos de los telescopios. Sin embargo, este procedimiento no es
viable con \MAGIC: simplemente hay demasiados espejos para
monitorizarlos a mano, por no hablar de la dificultad de acceso a
determinadas zonas del plato.  Un m'etodo alternativo ser'ia el uso de
una l'ampara de deuterio lejana. El haz de luz se har'ia pasar a
trav'es de un diafragma y se enfocar'ia en el telescopio. Con el
control activo de los espejos ser'ia posible hacer que la luz
reflejada en un espejo individual determinado incidiese en una
fotoc'elula o un fotodiodo para medir la cantidad de luz.  Tomando
medidas sucesivos d'ias y usando un espejo de referencia se podr'ia
estimar la degradaci'on de la reflectividad de los espejos del plato.

\subsubsection{Monitor del campo visual}
%
Debido a la gran superficie de espejos de \MAGIC, una cantidad
considerable de luz proviniente del campo de estrellas se concentrar'a
dando lugar a imagenes de estrellas en la c'amara.  Como resultado de
la utilizaci'on de una montura alt-azimutal, esta imagen del campo
estrellado rotar'a en el plano de la c'amara durante el seguimiento de
un objeto determinado durante la noche.  Existen dos posibles
procedimientos para paliar esta situaci'on: uno ser'ia eliminar de la
l'ogica de \emph{trigger} aquellos p'ixeles \emph{calientes}, es
decir, aquellos p'ixeles que reciben una cantidad de luz de las
estrellas del campo suficiente por si misma para inducir el sistema a
registrar un suceso ficticio; o bien, se puede disminuir el umbral de
disparo de los fotomultiplicadores (para minimizar la cantidad de
sucesos ficticios) y registrar las condiciones de la c'amara y del
campo visual, grabando las posiciones de las estrellas, con objeto de
deconvolucionar su efecto en los datos \emph{a posteriori}.  Para esto
es necesario la utilizaci'on de un monitor del campo visual.

Como efecto adicional, el monitor del campo visual podr'ia suministrar
datos al sistema de seguimiento e incluso a un sistema de autoguiado.

\subsubsection{Monitor de condiciones ambientales}
%
Existen numerosas fuentes de variabilidad de las condiciones
observacionales: no s'olo cambios meteorol'ogicos, sino tambi'en la
orientaci'on del telescopio, variaciones del contenido de aerosoles en
la atm'osfera, cantidad de luz del cielo nocturno, etc.  Estos
elementos afectan, entre otras cosas, a la transmisi'on atmosf'erica.
Para poder usar de manera conjunta observaciones de diferentes noches
(o incluso de una misma noche), necesitamos estudiar estas
variaciones.

La transmisi'on atmosf'erica se puede monitorizar de diversas maneras:
\begin{itemize}
\item Como ya se mencion'o m'as arriba, se prev'e utilizar una c'amara
  CCD en el sistema de seguimiento. Se podr'ia usar una segunda CCD
  para medir la calidad del cielo, dado que una cantidad apreciable de
  estrellas aparecer'ia en el campo visual del telescopio.
  
\item Se usar'a un peque~no telescopio 'optico para la cuantificaci'on
  de variaciones 'opticas de fuentes de rayos $\gamma$ que aparezcan
  pulsantes o que sufran estallidos.  Este telescopio podr'ia usarse
  tambi'en para la medida de las variaciones del brillo del cielo en
  la zona hacia donde \MAGIC est'e apuntando.
  
\item Finalmente, se planea la instalaci'on de un sistema LIDAR para
  registrar los perfiles de retro-dispersi'on de la luz en la
  atm'osfera.
\end{itemize}

%\section{Data Acquisition}
%

%\section{Observational Programs}
%

%\section{The Telescope CT1 of the HEGRA Collaboration}
%%


\endinput
%
%% Local Variables:
%% mode:latex
%% TeX-master: t
%% End:

%%EOF
