% +--------------------------------------------------+
% | Typeset this file to get the documentation.      |
% +--------------------------------------------------+
%
% This file is copyrighted (c) 1998 by Javier Bezos.
%
% This file is distributed without any warranty and specificallly
% disclaims any implied warranty of MERCHANTABILITY or FITNESS
% FOR ANY PARTICULAR PURPOSE. While the author firmly belives this
% to be a high quality product, the user must assume all risks of
% using the file.
%
% NO PERMISSION is granted to produce or to distribute a modified
% version of this file under its original name.
%
% You are NOT ALLOWED to change this file.
%
% For further copyright information contact with the author.

\def\fileversion{1.1}
\def\docdate{May 6, 1998}

\documentclass{ltxguide}
\usepackage[bf,sf,compact]{titlesec}


\title{The \textsf{titlesec} Package\footnote{This 
package is currently at version 1.1.}}

\author{Javier Bezos\footnote{Please, send your
comments and suggestions to \texttt{jbezos@mx3.redestb.es}, or
to my postal address: Apartado 116.035, E-28080 Madrid, Spain.
English is not my strong point, so contact me when you
find mistakes in the manual. Other packages by the same author:
\textsf{accents, polyglot, spanishb}.}}

\date{1998/3/6}

\setheadindent{2.1pc}{0pc}

\renewpagestyle{plain}[\small\sffamily\slshape]{
  \footrule
  \setfoot{}{\usepage}{}}

\newpagestyle{myps}[\small\sffamily\slshape]{
  \headrule
  \sethead{Titlesec}{\sectiontitle}{\usepage}}
  
\pagestyle{myps}

\addtolength{\topmargin}{-3pc}
\addtolength{\textwidth}{6pc}
\addtolength{\oddsidemargin}{-2pc}
\addtolength{\textheight}{7pc}

\raggedright
\parindent1em
\nofiles

\begin{document}

\maketitle

With this package, you can change in a straightforward way the 
sectioning format in a document.  It works with the standard classes 
and with many others (but not the AMS ones).  Because |\part| is usually 
implemented in a non-standard way, it remains untouched and should 
be changed ``by hand.''  (Not a real quibble since this command is not 
very used.)

The package is loaded by |\usepackage| and then you redefine the 
sectioning commands.  That may be done by the package itself (see 
section ``Simple Interface'') or by you with the provided commands (see 
section ``Advanced Interface.'')  In the latter case, you only need 
redefine the commands you'll use.  Both methods are available at 
time.

\section{Simple Interface}

The easiest way to change the format is by means of a set of package 
options.

\paragraph{Format}
There are three option groups controlling font, size and align. You 
need not set all of these groups, since a default is provided for
each one; however, you must at least set one of them, or otherwise 
\textsf{titlesec} performs no formatting.
\begin{decl}
|rm sf tt md bf up it sl sc|
\end{decl}
which select the corresponding family/series/shape.  Default is |bf|.  

Size adjustment is allowed with
\begin{decl}
|big medium small tiny|
\end{decl}
Default is |big|, which gives the size of standard classes.

Align is controlled by
\begin{decl}
|flushleft center flushright|
\end{decl}

\paragraph{Spacing}
This options is independent from the above and reduces the spacing.
\begin{decl}
|compact|
\end{decl}

For example, at the beginning of this document we just said
\begin{verbatim}
\usepackage[bf,sf,compact]{titlesec}
\end{verbatim}


\section{Advanced Interface}

Two commands are provided to change the title format. The first one is 
used for the ``internal'' format, i.~e., shape, font, label\dots, 
the second one defines the ``external'' format, i.~e., spacing before 
and after, indentation\dots{} This scheme is intended to easy definitions, 
since in most of cases you will want to modify either spacing or 
format.\footnote{Information is ``extracted'' from 
the class sectioning commands, except for chapter.  Standard 
definitions with |\string\@startsection| are presumed, and if an special 
procedure was followed an arbitary value is provided, which you may 
change later. Sadly, there is no way to fetch the chapter format and 
one similar to that of standard classes will be used.} Note that you
can redefine existing sectioning commands, but you cannot define \emph{new} ones.

\begin{decl}
|\titleformat{<command>}[<shape>]{<format>}{<label>}{<sep>}{<style>}|
\end{decl}

Here
\begin{itemize}
\item |<command>| is the sectioning command to be redefined, i.~e., 
|\chapter|, |\section|, |\subsection|, |\subsubsection|, |\paragraph| 
or |\subparagraph|.  As stated above, \textsf{titlesec} does not 
handle |\part|.

\item The paragraph shape is set by |<shape>|, which can
be:\footnote{Further styles are intended in future releases.}
\begin{description}
\item[hang] is the default value, with a hanging label.

\item[block] typeset the whole title in a block (a paragraph) without
additional formatting.

\item[display] puts the label in a separate paragraph.

\item[runin] A run-in title. (Like the standard |\paragraph|.) Not 
available to |\chapter|.

\item[leftmargin] puts the 
title at the left margin.  Large titles at the very end of a page can lead 
to unpredictable results, but short titles work pretty well and do not
stick out in the bottom margin. Not available to |\chapter|. |margin| 
is a synonymous.

\item[rightmargin] is like |leftmargin| but at the right margin.

\item[drop] wraps the text around the title.  The comment in |leftmargin| also 
applies here. Not available to |\chapter|.
\end{description}

\item |<format>| is the format to be applied to the whole 
title---label and text.

\item The label is defined in |<label>|.  See below for illustrative 
examples.

\item |<sep>| is the separation between label and title body.  This 
space is horizontal in |hang|, |block|, |runin|, |drop| and |...margin|, and
vertical in |display|. Both |<label>| and |<sep>| are ignored in
starred versions of sectioning commands.

\item And finally, |<style>| is the style of the title body. The 
very last command can take an argument, which is the title text.
\end{itemize}

\begin{decl}
|\titlespacing*{<command>}{<left>}{<before>}{<sep>}|
\end{decl}

The starred version supresses the indentation of the paragraph 
following the title, except in |drop| and |runin| where this
alternative does not make sense.
\begin{itemize}
\item |<left>| is the ``left space'' of the title.  In the |...margin|
and |drop| shapes this parameter set the title width, in the |runin| one the
indentation, and in the remainder a new left margin with
respect to the current one. With negative value the title 
hangs. This parameter is not equal to |<indent>| of 
|\@startsection|, which doesn't work correctly.
With a negative value in the latter, the first line of the title will
start in the outer margin, as expected, but the second, third\dots{} 
lines will not; worse, those lines will be shortened by |<margin>|.

\item |<before>| is the vertical space before the title.
\item |<sep>| is the separation of title and text---vertical with |hang|,
|block|, and |display|, and horizontal with |runin|, |drop| and |...margin|.
By making the value negative, you may define an effective space
of less than |\parskip|.\footnote{See Goosens, Mittelbach and
Samarin: \textit{The \LaTeX{} Companion}, Reading, Addison Wesley, 
1993, p.~25.}
\end{itemize}

\begin{decl}
|\filright  \filcenter  \filleft  \fillast  \filinner  \filouter|
\end{decl}

Drastic versions of the |\ragged...| commands, which cancel word 
hypenation.  |\fillast| justifies the paragraph, except the last line 
which is centered.  If the document is |\raggedright| you must reset 
the value of |\rightskip| to 0 pt, or some irregular format can arise.

|\filinner| and |\filouter| are |\filleft| or |\filright| depending
on the page. Because of the asyncronous \TeX{} page 
breaking, these commands can be used in |\chapter| only.

\begin{decl}
|\wordsep|
\end{decl}

The inter-word space of the current font. See below for an example.

Related package options are:
\begin{decl}
|indentfirst nonindentfirst|
\end{decl}

By-pass the settings for all of sectioning command.

\begin{decl}
|rigidchapters rubberchapters|
\end{decl}

With |rigidchapters| the space for chapter titles is always the same 
and |<sep>| in |\titlespacing| is the total amount reserved to them, 
i.~e., |<before>| $+$ |<sep>| is the distance
from the top of the page to the
main text.  The default is |rubberchapters| which |<sep>| is the 
separation of title and text in, as usual.

\begin{decl}
|clearempty|
\end{decl}

Modifies the behaviour of |\cleardoublepage| so that the |empty| page style
will be used in empty pages.

The following examples will be illustrative.

\begingroup

\rightskip=0pt

\begin{verbatim}
\titleformat{\section}[block]
  {\normalfont\bfseries\filcenter}{\fbox{\itshape\thesection}}{1em}{}
\end{verbatim}

\titleformat{\section}[block]
  {\normalfont\bfseries\filcenter}{\fbox{\itshape\thesection}}{1em}{}

\section[Advanced Interface]{This is an 
example of the section command just defined and, what's more, this is 
an example of the section command just defined}

\begin{verbatim}
\titleformat{\section}[display]
  {\normalfont\narrower\narrower\fillast}
  {\scshape section \oldstylenums{\thesection}}
  {1ex minus .1ex}
  {\small}
\end{verbatim}

\titleformat{\section}[display]
  {\normalfont\narrower\narrower\fillast}
  {\scshape section \oldstylenums{\thesection}}
  {1ex minus .1ex}
  {\small}

\section[Advanced Interface]{This is an example of the section
command just defined 
and, what's more, this is 
an example of the section command just defined. Let us repeat it.
This is an example of the section command just defined 
and, what's more, this is 
an example of the section command just defined}

\begin{verbatim}
\titleformat{\section}
  {\hrule\vspace{.5ex}\normalfont\itshape}
  {\thesection.}{.5em}{}
\end{verbatim}

\titleformat{\section}
  {\hrule\vspace{.5ex}\normalfont\itshape}
  {\thesection.}{.5em}{}

\section[Advanced Interface]{A Ruled Title}

\begin{verbatim}
\titleformat{\section}[block]
  {\normalfont\sffamily}
  {\thesection}{.5em}{\hrulefill\\[.8ex]\bfseries}
\end{verbatim}

\titleformat{\section}[block]
  {\normalfont\sffamily}
  {\thesection}{.5em}{\hrulefill\\[.8ex]\bfseries}

\section[Advanced Interface]{Another Ruled Title}

\begin{verbatim}
\titleformat{\section}[runin]
  {\normalfont\scshape}
  {}{0pt}{}
\titlespacing{\section}
  {\parindent}{\medskipamount}{\wordsep}
\end{verbatim}

\titleformat{\section}[runin]
  {\normalfont\scshape}
  {}{0pt}{}
\titlespacing{\section}
  {\parindent}{\medskipamount}{\wordsep}
  
\section*{The investigation of transport processes}
requires a consideration of the effects of molecular\ldots

\begin{verbatim}
\titleformat{\section}[drop]
  {\normalfont\fontseries{b}\selectfont\filright}
  {\thesection.}{.5em}{}
\titlespacing{\section}
  {12pc}{1.5ex plus .1ex minus .2ex}{1.5pc}
\end{verbatim}

\titleformat{\section}[drop]
  {\normalfont\fontseries{b}\selectfont\filright}
  {\thesection.}{.5em}{}
\titlespacing{\section}
  {12pc}{1.5ex plus .1ex minus .2ex}{1.5pc}

\section[Advanced Interface]{A Simple Example of the
  ``drop'' Section Shape}

Which is followed by some text to show the result. And blah, blah, blah, 
blah,  blah,  blah,  blah,  blah,  blah,  blah,  blah,  blah,  blah, 
blah,  blah,  blah,  blah,  blah,  blah,  blah,  blah,  blah,  blah, 
blah,  blah,  blah,  blah,  blah,  blah,  blah.  And blah, blah, blah, 
blah,  blah,  blah,  blah,  blah,  blah,  blah,  blah,  blah,  blah, 
blah,  blah,  blah,  blah,  blah,  blah,  blah.

\begin{verbatim}
\titleformat{\section}[leftmargin]
  {\normalfont\sffamily\bfseries\filleft}
  {\thesection}{.5em}{}
\titlespacing{\section}
  {4pc}{1.5ex plus .1ex minus .2ex}{1pc}
\end{verbatim}

\titleformat{\section}[leftmargin]
  {\normalfont\sffamily\bfseries\filleft}
  {\thesection}{.5em}{}
\titlespacing{\section}
  {4pc}{1.5ex plus .1ex minus .2ex}{1pc}

\section*{Example of margin section}

Which is followed by some text to show the result.
But don't stop reading, because the following example illustrates how
to take advantage of other packages. The last command in the last
argument can take an argument, which is the title with no
aditional command inside it. We just give the code, but you
may try it by yourself. Thus, with the \textsf{letterspace} package
you may say
\begin{verbatim}
\newcommand{\secform}[1]{
  \MakeLowercase{\protect\letterspace to 1.3\naturalwidth{#1}}}
\titleformat{\section}[block]
  {\normalfont\scshape\filcenter}
  {\thesection}
  {1em}
  {\secform}
\end{verbatim}

\endgroup

\smallskip
\begin{center}
*\hspace{3pc}*\hspace{3pc}*
\end{center}
\bigskip

\section{Page Styles}

As a part of this package a set of commands are provided for one-step
headline and footline settings.  It's not intented for complex headlines 
which packages like \textsf{fancyhdr} o \textsf{scrpage}
are best suited for, but most of 
cases can be handled. The main limitation is that you must restrict 
yourself to pass information to headlines; footlines may contain text 
and page numbers, but not section titles, numbers, etc. It works with
standard classes and others, even if titles are not redefined;
however, it clashes with the \textsc{AMS} classes, which provide
their own solutions. If you use an alternative package, like those
above, don't forget the package option
\begin{decl}
|nops|
\end{decl}
and all of commands described in this section will be ignored. 

\begin{decl}
|\newpagestyle{<name>}[<global-style>]{<commands>}|\\
|\renewpagestyle{<name>}[<global-style>]{<commands>}|
\end{decl}

Defines a new style and redefines an existing one, named |<name>|.  For 
convenience, end of lines are ignored, and you needn't hide them with 
|\%|.

|<global-style>| is any command to be applied to both footlines and 
headlines. Fragile commands are allowed.

In |<commands>| you may use
\begin{decl}
|\headrule \footrule|
\end{decl}

If you want a rule below the headline and above the footline.  You may 
also set their width directly with the commands |\setheadrule| and 
|\setfootrule|. (For instance, |\setheadrule{.4pt}| which is,
by the way, the default value.)

\begin{decl}
|\sethead[<even-left>][<even-center>][<even-right>]|\\
|        {<odd-left>}{<odd-center>}{<odd-right>}|\\
|\setfoot[<even-left>][<even-center>][<even-right>]|\\
|        {<odd-left>}{<odd-center>}{<odd-right>}|
\end{decl}

Set the corresponding part in heads and foots.  The optional arguments 
are all or none.  If no optional arguments are given the odd-page 
settings applies to both even and odd pages.  In the arguments of 
|\sethead| and because of its one-step mechanism, we must distingish 
between two groups of command. The first one are those expanded when 
the marks are stored, with information related to sections, and 
includes:
\begin{itemize}
\item |\thechapter|, |\thesection|\dots{}
\item |\chaptertitle|, |\sectiontitle|\dots{} which print the chapter, 
section\dots{} title.
\item Any command which 
is neither fragile nor protected (a command just containing text, for 
example).
\end{itemize}
The second one are those expanded in every page, and includes:
\begin{itemize}
\item |\usepage|.  The 
page number, but evaluated in every page.  (|\thepage| is evaluated 
when the mark is sent---not a desired behaviour, indeed.)
\item Any protected command.
\end{itemize}
Fragile commands must be protected with |\protect|.

In |\setfoot| the commands are always evaluated in every page.

\begin{decl}
|\setmarks{<primary>}{<secondary>}|
\end{decl}
Sets which |\...title| commands are to be defined, and when the marks
are updated. For example 
|\setmarks{chapter}{section}|
means that:
\begin{itemize}
\item |\chaptertitle| and |\sectiontitle| are the allowed titles in 
  heads,
\item |\sectiontitle| is reset in |\chapter|, and
\item 
  |\markboth{<even-head>}{<odd-head>}| is executed in |\chapter| and 
  |\section|, where |<even-head>| and |<odd-head>| stands for the 
  internal definitions generated by |\sethead|.
\end{itemize}
By default it is 
|\setmarks{chapter}{section}| but |\setmarks{section}{subsection}| in 
\textsf{article} class.

Related package options are:
\begin{decl}
|outermarks innermarks topmarks|
\end{decl}

|innermarks| is the default in \LaTeX, with bottom marks in even pages 
and first marks in odd pages---not useful, not usual and, furthermore, it
can lead to undesired results.  More 
convenient is |outermarks|, with top marks in even pages and bottom 
marks in odd pages; it's the usual system and described in \textit{The 
\TeX{}book}, p.~259.  Both of them are intended for two side printing; 
|topmarks| is intended for one side printing with top marks in every 
page.

Since |\markboth| is used when the marks are updated, as set by |\setmarks|,
special constructions like that are allowed:
\begin{verbatim}
\newpagestyle{mypagestyle}[\small\sffamily]{
  \sethead [\textbf{\usepage}] 
           [\textsl{\chaptertitle}]
           [[\thesection]
           {\thesection]} 
           {\textsl{\sectiontitle}}
           {\textbf{\usepage}}}
\end{verbatim}
As you can see, |\thesection| is printed in both left and right heads (with
right values!)

\begin{decl}
|\setheadindent[<even-left>][<even-right>]{<odd-left>}{<odd-right>}|
\end{decl}

Sets the indent of footlines and headlines.

In this document, the following styles are used:
\begin{verbatim}
\setheadindent{2.1pc}{0pc}
\renewpagestyle{plain}[\small\sffamily\slshape]{
  \footrule 
  \setfoot{}{\usepage}{}}
\newpagestyle{myps}[\small\sffamily\slshape]{
  \headrule
  \sethead{Titlesec}{\sectiontitle}{\usepage}}
\end{verbatim}

The following definitions provide page styles similar to
those of Lamport's \LaTeX\ book.

\begin{verbatim}
\newlength{\headind}
\setlength{\headind}{\marginparsep}
\addtolength{\headind}{\marginparwidth}
\setheadindent[\headind][0pt]{0pt}{\headind}
\renewpagestyle{plain}{}
\newpagestyle{latex}[\bfseries]{
  \headrule
  \sethead[\usepage][][\chaptertitle]
          {\thesection\ \sectiontitle}{}{\usepage}}
\end{verbatim}

\section{Example}

Now an example of a complete new title scheme follows.

\begin{verbatim}
\documentclass[twoside]{report}
\usepackage[sf,sl,outermarks]{titlesec}

% \chapter, \subsection...: no additional code

\titleformat{\section}
  {\LARGE\sffamily\slshape}
  {\thesection}{1em}{}
\titlespacing{\section}
  {-6pc}{3.5ex plus .1ex minus .2ex}{1.5ex minus .1ex}

\titleformat{\paragraph}[leftmargin]
  {\sffamily\slshape\filright}
  {}{}{}
\titlespacing{\paragraph}
  {5pc}{1.5ex minus .1 ex}{1pc}

% 5+1=6, ie, the negative left margin in section

\setheadindent{6pc}{0pc}
  
\renewpagestyle{plain}{}

\newpagestyle{special}[\small\sffamily]{
   \headrule
   \sethead[\usepage][\textsl{\chaptertitle}][]
           {}{\textsl{\chaptertitle}}{\usepage}}
               
\newpagestyle{main}[\small\sffamily]{
   \headrule
   \sethead[\usepage][\textsl{\thechapter. \chaptertitle}][]
           {}{\textsl{\thesection. \sectiontitle}}{\usepage}}

\pagestyle{special}

\begin{document}

---TOC

\pagestyle{main}

---Body

\pagestyle{special}

---Index
\end{document}
\end{verbatim}


\section{Other Changes}

In addition, this version has the following changes:
\begin{itemize}
\item The |\chapter| as generated by the ``simple interface'' has been
modified, since version 1.0 didn't reproduce correctly the standard 
classes one. Further, space between label and body now depends on the 
font size.

\item Instead of the promised |mirrorchapters| option, I provide the
|\filinner| and |\filouter| commands.

\item The starred version of |\titlespacing| now works in |...margin|.
\end{itemize}

\section{To do}
\begin{itemize}
\item Improved |...margin| title styles.  
\item Package options |bottomtitles| and |nobottontitles|.
\item Perhaps |\newtitle|, |\newtitlestyle|, and |\newchapterstyle|.
\end{itemize}

Happy \TeX{}ing!
\end{document}